\chapter{O ano em que o mundo foi salvo}

Jonas tinha um apetite pantagruélico por respostas. Ele não conseguiria dar uma resposta racional para explicar sua motivação. \mudanca{Se} sentia impelido a buscar mais informações. Edgar, Arthur e Frederico, ou Os Renascidos, era um grupo informal de estudos sobre \textsc{eqm} \mudanca{que} se \mudanca{reunia} sempre, seja na Novo Caminho ou em um lugar ali próximo, chamado Café Filosófico, para discutir sobre as experiências de quase-morte. Por mais que o relacionamento com Falls estivesse dando certo, ele ainda tinha profundas inquietações em sua alma. Uma fome insaciável por respostas. E foi atrás delas que ele passou regularmente a se reunir com eles.

--- É um campo vasto, Jonas. Diz muito sobre o grande enigma do mundo. Mas, diga, por que você se sente tão atraído por isso? --- lhe disse Edgar\mudanca{,} com o olho já um pouco melhor. Não estava mais tão roxo como antes, mas se via uma pequena sombra, como pegadas na areia.

--- Desde que\ldots\,--- Jonas simplesmente não conseguia pronunciar as palavras que estavam logo ali em seus pensamentos. Eles sempre estavam aqui. Seja na superfície ou ali no fundo, eles sempre o acompanham aonde quer que ele vá, está nele. Em seus genes. Como uma bomba-relógio colocada em sua nuca. --- Eu não sei racionalizar minha fascinação. Minha mãe foi espírita por algum tempo. Deve ter vindo daí. Vocês devem ter centenas de histórias interessantes.

--- Esse é o nosso estilo de vida, Jonas --- falou Arthur\mudanca{m} rindo.

--- Vocês possuem teorias sobre as \textsc{eqm}s? --- Jonas \mudanca{perguntou ansiosamente}.

--- Eu acho que cada um de nós possui sua própria teoria --- falou Frederico. --- Ou teorias. Mas tudo muito inconclusivo.

--- Você sabia que\mudanca{,} em 1988, pessoas de diversos lugares do mundo, sem se conhecerem previamente, voltaram de estados de quase-morte com profecias de que o mundo iria acabar? --- lhe disse Arthur.

--- Talvez algum espírito do Outro Lado tenha pregado uma peça, ou algo assim --- Edgar notou.

--- Ou quem sabe, alguém salvou o mundo --- Jonas sugeriu\mudanca{,} a título de piada. Quem sabe? Depois do Big Bang\mudanca{,} nada mais é impossível.

--- Jonas, eu vou \mudanca{te} fazer uma pergunta\mudanca{,} e quero que me responda com sinceridade --- Edgar começou a dizer\mudanca{,} calculadamente.

--- Claro --- disse Jonas.

--- Diga-nos como foi a sua \textsc{eqm}. Você pode se abrir. Está entre amigos aqui.

--- Eu já disse --- Jonas ficou sem jeito, começando a tropeçar nas palavras. --- Eu\ldots\,Bem, eu somente sentia dor, a asfixia, e nada. Tenho alguns fragmentos que duram nanossegundos\mudanca{, em} minha memória. Como um \mudanca{lapso}. E ampolas de ``Ketanest~S'' e ``Nivalin''. É tudo o que eu lembro.

--- Você disse ``Ketanest~S'' e ``Nivalin''?

--- Sim. Por quê? --- Jonas perguntou.

--- Apenas curiosidade --- Frederico acrescentou.

Eles o olhavam tentando adivinhar se ele falava ou não a verdade.

--- Sabe por que fiz a pergunta? --- Edgar perguntou. --- Nem todas as \textsc{eqm}s são prazerosas. Nem todos vão para um lugar agradável, como muitos relatam. Alguns tem \textsc{eqm}s terríveis. É apenas o reflexo de uma mente atormentada pela culpa, ou algum trauma. Alguns relatam lugares como o Inferno, sabe? E geralmente por medo, culpa ou vergonha eles não contam para ninguém. Guardam essa coisa para si. Não que duvide de sua palavra, mas eu tinha que me certificar, sabe? Se fosse o caso talvez pudéssemos ajudar.

--- Já assistiu \emph{O sexto sentido\/}? --- perguntou Frederico para Jonas após a mesa ter sido dominada pela reflexão de cada um dos ocupantes.

--- Já, claro --- \mudanca{respondeu}.

--- Em uma cena chave, o garoto conta que ele vê gente morta. E que a maioria delas não sabe que está morta. A maioria delas. Onde eu quero chegar é que a \textsc{eqm} me ensinou que a maioria das pessoas não sabe que está viva. E isso acontece todo o tempo. Saber que está vivo é ter a consciência de que a vida não é algo consolidado, sólido, mas algo destinado a acabar. A qualquer momento. Por isso elas precisam tomar conta de suas vidas e fazer \mudanca{valer} a pena. Mas elas continuam em letargia, inertes, no piloto automático. Eu tive muito medo de dar o primeiro salto. Eu acordei, Jonas.

--- Como foi?

--- Quando eu fiz a minha primeira, ainda não existia a Novo Caminho. Fui a terceira cobaia. Era amigo de Arthur e ele me confidenciou a transformação que ocorrera em sua vida. Eu queria aquilo para mim. Também, quem não gostaria? Pois bem, após sentir uma dor terrível em meu peito e me sentir como se estivesse afogando, desesperadamente, eu me vi flutuando, me sentindo feliz, e vendo um filme em visão \mudanca{de trezentos e sessenta graus}, todas as direções ao mesmo tempo! Que bela visão era aquela\ldots\,Começou naquele dia e foi indo em marcha a ré, era muito rápido, mas conseguia ver todos os momentos de minha vida, com cheiros e sabores\mudanca{. Era} sensorial. Quando chegou no dia do meu nascimento\mudanca{,} vi-me transportado\mudanca{,} sem saber como, para um túnel que girava mas que não me deixava tonto. O túnel parecia ser feito de nuvens, mas \mudanca{que} mudavam de cor o tempo todo\mudanca{,} e era imenso em largura e altura. Pessoas andavam neste túnel, parecendo deslizar, e eu deslizava também, em direção a uma luz que havia no fundo, muito dourada, mas que não cegava a vista. Era uma luz aconchegante. Eu ia indo, sem nem entender \mudanca{por que}, mas me sentia feliz. Então alguém\mudanca{,} que me pareceu um homem, \mudanca{tinha} a cabeça envolta num manto, parecido mais com um capuz marrom, me chamou pelo meu nome e disse: ``Você não era para estar aqui, mas\mudanca{,} se quiser\mudanca{,} pode continuar''. Eu perguntei: ``E se eu não continuar, o que acontece?'' Ele respondeu que eu iria viver a vida para a qual tinha vindo. Eu hesitei um pouco e ele\mudanca{,} então\mudanca{,} fez um gesto que me pareceu alguém abrindo uma cortina de nuvens\mudanca{,} e \mudanca{vi} lá embaixo \mudanca{pessoas ao} redor do meu corpo flácido, sem vida. Vi meu corpo arroxeado. No instante seguinte\mudanca{,} o túnel tinha desaparecido, e eu estava de volta ao meu corpo, sentindo-me muito mal, e surdo. Minha surdez passou gradualmente, em pouco mais de \mudanca{quinze} dias, e comecei a sonhar com o homem do capuz todas as noites por \mudanca{três} anos. Estabelecemos uma relação de amizade, e ele disse que estaria comigo durante toda a minha vida. Nunca esquecerei, e o que mais me impressionou foi o filme de todas as minhas recordações\mudanca{;} ainda mais: ``sensorial!'' Isto levou-me posteriormente a estudar física e psicologia, mas nunca levou-me a nenhuma religião. Perdi todo o medo da morte, porque sei que morrer seria passar por aquilo, o que não era\mudanca{,} absolutamente\mudanca{,} em nada ruim. Já saltei outras vezes, depois que o homem com o capuz deixou de me visitar.

Jonas ouviu aquilo com um misto de admiração e inveja. Em sua experiência tudo o que ele sentiu foi o desconforto e a dor\mudanca{, acordando} como se nada tivesse acontecido.

--- Quem você acha que é o sujeito do capuz?

--- Não tenho ideia. Talvez o arquétipo do pai ou, quem sabe\mudanca{?,} Deus. Poderia até mesmo ser Elvis.

--- Elvis?

--- Essa é uma piada interna, Jonas --- Arthur o alertou. --- Se você estudar relatos de \textsc{eqm}s\mudanca{,} descobrirá que muitas delas relatam encontros com figuras conhecidas como Jesus, Buda. O que não seria de estranhar. Mas há algumas pessoas que dizem ter encontrado Elvis. Na Novo Caminho\mudanca{,} há poucos meses\mudanca{,} tivemos um caso do tipo. Um sujeito disse ter visto John Lennon em sua \textsc{eqm}.

--- Sério? E o que John lhe disse?

--- ``Paz. A mensagem é Paz'', foi o que ele lhe disse. Ele passou a vir em nossas conversas. Ele dizia que as músicas dos Beatles continham os segredos do Universo, a resposta para todas as perguntas.

--- Isso eu acho que diz muito sobre tudo. Quero dizer, há bilhões de pessoas no mundo. Mesmo que elas frequentem as mesmas religiões, torçam para os mesmos times\mudanca{,} em uma ou outra coisa \mudanca{têm} pensamentos diferentes. Temos bilhões de indivíduos, bilhões de concepções diferentes do que é a realidade.

--- Antes mesmo de fazer o primeiro salto, eu estudei sonhos lúcidos. Na verdade\mudanca{,} conheci o Roberto lá. E, muita coisa que as pessoas experimentam nas \textsc{eqm}s, você consegue de alguma forma emular, com muito treinamento, em um sonho lúcido --- falava Arthur\mudanca{,} gesticulando muito. --- Visão \mudanca{de trezentos e sessenta graus}, experiência fora do corpo. Isso me leva a pensar, sabem. O que é a realidade? Sabem? Não seria uma espécie de sonho coletivo? Ou algo assim? \mudanca{Porque} eu já tirei sonecas de\ldots\,Digamos, cinco minutos, certo? E tive um sonho que apareceu durar horas, senão dias. É uma coisa muito louca. Em um dos meus saltos, o quinto ou o sexto, eu experimentei a sensação mais incrível de toda a minha vida, Jonas. Eu já enchi o saco desses dois contando isso, mas você ainda não sabe. Durante esta experiência, tempo não tinha significado. Tempo era uma noção irrelevante. Eu senti a eternidade. Eu senti como se houvesse uma eternidade. Deus não promete eternidade? Eu experimentei isso.

--- Uau --- disse Jonas. --- Eu nem consigo pensar direito nisso.

--- Todos os outros saltos que eu fiz após aquele foram para conseguir aquilo de novo. É como se nossa realidade e essas noções de espaço-tempo fossem prisões. Como o mito da caverna de Platão.

--- Então vocês acreditam em alma? --- Jonas perguntam.

--- Depende do que se quer dizer com alma --- Arthur falou. --- A vida é cheia de possibilidades. Eu sei que há algo além da nossa existência ordinária, algo que somente com nossa morte física temos acesso. Tudo mais a respeito disso, deixo para os pensadores como Edgar.

--- E mesmo eu não tenho conclusões sólidas --- Edgar disse\mudanca{,} sem falsa modéstia. --- Meu pai sempre dizia, ``crê naquele que tem dúvida, duvida daquele que não as tem''. Uma atitude agnóstica não apenas para Deus, mas para qualquer assunto. Indo do átomo até mesmo ao ornitorrinco. Eu acredito que haja uma pequena glândula em nossos cérebros que está relacionada a diversos desses fenômenos. Sonhos lúcidos, experiências extracorpóreas, telepatia, telecinese\ldots

--- Glândula?

--- Chama-se glândula pineal. Mas\mudanca{,} como eu disse, não há certeza de nada. \mudanca{Na} Novo Caminho eu instalei um gerador de imagens e números aleatórios, quem reportar experiências extracorpóreas associadas à \textsc{eqm} poderá nos fornecer indícios. Antes as \textsc{eqm}s aconteciam naturalmente, um estudo sistemático era quase impossível. Agora, com o \emph{Método Ars Moriendi}\mudanca{,} poderemos torná-lo viável.

--- E como vão indo os resultados?

--- Ainda estamos engatinhando. Temos que ter um grande volume de dados antes de publicar qualquer coisa. A comunidade científica é muito fechada. Os cientistas se agarram a seus dogmas. Se você dá um passo em falso, eles te apontam o dedo e gritam: ``Blasfêmia!'' Mas há um grande volume de dados coletados. \textsc{eqm}s existem. Isso não se pode negar. É fato. A alegação de que não há consciência após a morte do cérebro não é baseada em experiência alguma. É apenas uma hipótese com base em suposições de que um dia os cientistas poderão explicar toda a atividade mental em termos de atividade cerebral. Mas isso é uma aposta, é fé, e não ciência. A maioria de nós tem certas crenças sobre a morte, mas, racionalmente, temos que admitir que não sabemos nada sobre ela, nem os cientistas. E nós tememos aquilo que desconhecemos. É difícil aceitar calmamente algo que tememos. Conforme Shakespeare afirmou no famoso solilóquio de Hamlet, é isso que ``nos faz suportar os males que possuímos, em vez de partirmos para terras não-descobertas da qual viajante nenhum jamais retornou.'' Na verdade\mudanca{,} os relatos apontam para o inverso disso. E com o desenvolvimento e refinação de métodos de ressuscitação, alguns viajantes estão podendo retornar. Um caso histórico no caminho de demonstrar sua validade é o de Pam Reynolds.

--- O que aconteceu com essa Pam? --- Jonas disse\mudanca{.} se ajeitando na cadeira e ansioso por saber mais.

--- Pam Reynolds é o nome artístico de uma cantora americana. Em 1991, com 35 anos de idade, ela teve uma experiência de quase-morte durante uma operação cerebral. Seu caso é um dos mais notáveis e melhores documentados por causa das circunstâncias em que ele aconteceu. Reynolds estava sobre monitoramento médico durante toda a operação. Durante parte de sua operação, ela não teve atividade cerebral por que não havia fluxo de sangue no cérebro. Ela fez muitas observações sobre o procedimento que mais tarde foram confirmadas pelos presentes como surpreendentemente corretas. Descobriram que havia um grande aneurisma cerebral em Pam Reynolds e o médico tomou a decisão de realizar um ousado procedimento\mudanca{,} para a época. Durante esta operação eles abaixam a temperatura do paciente para \mudanca{dezesseis graus centígrados}, sua respiração e batimentos cardíacos são parados e todo o sangue é drenado de sua cabeça. Os olhos são tapados com uma fita e pequenos protetores de ouvidos são colocados no paciente. Nesses protetores haviam pequenos reprodutores, como fones de ouvido, que emitiam cliques audíveis que seriam utilizados para testar as funções cerebrais. A operação foi um sucesso e ela se recuperou completamente. E mais, voltou da operação com descrições vívidas de tudo o que acontecia com seu corpo, inclusive presenciando quando seu crânio foi aberto. Inclusive\mudanca{,} ela foi capaz de \mudanca{repetir} uma frase que ouviu da equipe médica e que mais tarde foi confirmada como realmente proferida. Foi: ``Temos um problema. Suas artérias são bem pequenas''. Inclusive tocaram uma música enquanto ela estava fazendo a operação. Música \mudanca{que} ela soube dizer qual era. \foreignlanguage{english}{\emph{``Hotel California''}}, dos Eagles. Além da experiência-fora-corpo\mudanca{,} ela também se encontrou com entes queridos\mudanca{,} que a ajudaram. Mas, o debate continua.

--- O debate expõe dúvidas --- Frederico disse\mudanca{,} com um pequeno sorriso de satisfação.

--- Qual sua opinião, como um psicólogo? --- Jonas perguntou para Edgar.

--- Jonas, eu já tenho a minha opinião, mas como um ser humano. Eu não gosto quando alguém aponta e diz: ``Lá vai um empresário'' ou ``Olhe, aquele cara é um lixeiro''. Essa é simplesmente a forma \mudanca{como} o capitalismo vê um individuo\mudanca{:} uma engrenagem produtiva no sistema. Seja ele um lixeiro ou um psicólogo. As profissões são o que se faz para conseguir pagar o aluguel, mais nada. Não dizem absolutamente nada sobre você ou sua inteligência ou capacidade. No capitalismo, cada pessoa, $A$, é igual a uma produção, $B$, que rende um certo dinheiro, $C$. Então $A = B = C$. E como a matemática permite deduzir, se $A = B = C$, então $ A = C$. Então as pessoas têm um valor, são mensuráveis em dinheiro. Isso é algo grave, terrível\mudanca{,} para falar a verdade. Me desculpe por corrigi-lo nessa coisa mínima, \mudanca{sou} meio rabugento\mudanca{,} assim mesmo. Me desculpe.

--- Não, tudo bem. Na verdade, você está certo.

--- Eu já escrevi um livro. Um único livro. Chama-se \emph{Mors Ontologica}. Eu posso te emprestar uma cópia, se estiver interessado em saber minha tentativa de entender e racionalizar o assunto.

--- Obrigado --- disse Jonas\mudanca{,} olhando para o relógio. --- Eu tenho que ir. Tenho que buscar minha namorada.

--- No final\mudanca{,} o que todo mundo quer é apenas ser feliz --- Arthur disse para ele.

--- Ou como diria aquele nosso amigo, \foreignlanguage{english}{\emph{``All You Need Is Love''}}.

Jonas se despediu de cada um deles. Pagou sua conta e saiu em busca da amada.
