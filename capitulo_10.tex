\chapter{Dança macabra}

Três dias se passaram desde o salto de Jonas. Ele havia contado a George sobre sua experiência e ouvido\mudanca{,} novamente\mudanca{, todo aquele} discurso sobre roleta russa. \mudanca{Acabou} deixando de contar sobre Falls até o dia em que finalmente saíram\mudanca{,} pois tinha certeza \mudanca{que} George o atormentaria \mudanca{a} cada um dos dias anteriores, o que somente aumentaria \mudanca{sua} ansiedade e faria dar tudo errado.

Estava apenas tentando não errar. Contenção de riscos. Naquele terceiro dia, Jonas foi de bicicleta para o trabalho. George havia montado uma pequena tabela de coisas que eles teriam que fazer se quisessem compensar a sua fatia de poluição no planeta. Sustentabilidade pessoal. \mudanca{Seu} carro ainda era bem poluente\mudanca{,} um problema que ele tinha que lidar. Chegou todo suado.

--- Estou suado como um porco --- Jonas reclamou, ofegante, ao se sentar.

--- Tshh --- George desdenhou. --- Porcos não suam, idiota. Esta é a razão pela qual eles chafurdam na lama.

--- Tudo bem. Estou tão suado quanto uma prostituta em um bom dia de trabalho --- Jonas retrucou.

--- Ou noite --- George acrescentou.

--- Ou noite --- Jonas disse.

--- Eu vou precisar usar mais o carro, então\mudanca{,} para balancear\mudanca{,} estou vindo de bicicleta. Quer saber por quê? Porque eu convidei uma garota para sair e ela disse sim --- Jonas falou\mudanca{,} orgulhoso.

--- Mentira! --- George disse, rindo. --- Ela é cega?

--- Não é. Eu tenho até uma foto dela para provar.

--- O que não fazemos por nosso planeta? --- George falou\mudanca{, olhando} para a foto no celular de Jonas e fazendo uma careta ao reconhecê-la, mudando de assunto. --- O que nós precisamos é de uma nova praga. Há muitas pessoas por aí. Bilhões delas. \mudanca{A Terra} até que é um lugarzinho agradável, pena que seja muito mal frequentado.

Jonas apenas arqueou as sobrancelhas com a opinião de George.

--- Então, onde você vai \mudanca{levá-la}? --- George o questionou.

--- Eu não faço ideia. Acho que vou deixar ela decidir isso --- Jonas disse\mudanca{,} sem jeito.

--- Ela vai achar que você não é um homem de opinião. Então\mudanca{,} logo\mudanca{,} logo, \emph{wa-pah}! --- George disse\mudanca{,} fazendo um estranho movimento com a mão.

--- O quê? --- Jonas perguntou\mudanca{,} sem entender.

--- \emph{Wa-pah}! --- repetiu George tanto o som como o movimento com a mão.

--- O que isso significa? --- Jonas perguntou\mudanca{,} confuso, sem entender do que se tratava.

--- Um chicote, Jonas. \emph{Wa-pah}! --- falou George refazendo o movimento.

--- Isso não se parece com um chicote.

--- Não importa. Só não caia nesse erro. Eu mesmo\ldots\,Conheço um sujeito que caiu nisso.

--- Quem? --- Jonas achou estranho.

\mudanca{--- Irrelevante.}

--- Se você conhece, provavelmente \emph{eu} também conheço.

--- Não. Esse você não conhece.

--- Como ele se chama?

--- O que isso importa se você nem mesmo o conhece?

--- Eu perguntei por perguntar. Por que você não pode me dizer? É segredo?

--- Segredo? Quem falou em segredo? Ninguém falou em segredo! --- George parecia assustado.

--- Calma George.

--- É Jorge o nome dele. Com ``J'', como São Jorge.

Jonas achou aquilo tudo muito estranho, mas resolveu se concentrar em seu trabalho. Ele estava tão em paz, se sentindo tão confiante e bem consigo mesmo\mudanca{,} que resolveu nem dar bola para as implicações que George teve com ele durante o dia. Essa era a amizade deles. Eles sabiam que poderiam desafiar, pregar peças e até mesmo ofender um ao outro, que no final não havia ressentimentos. Eles se divertiam para valer. Mas Jonas sentia falta de uma profundidade maior em sua relação. Eles não contavam o que sentiam. Gastavam seu tempo falando de garotas que nunca iam conseguir, davam notas de zero a dez a todas que cruzavam, mas nunca tinham conversas sobre si mesmos. Jonas sentia falta de disso, mas não sabia nem mesmo por onde começar. Na hora da saída, George deu mais alguns conselhos para Jonas:

--- Eu sei o que deve estar pensando. Mas, não! Nada de flores ou bombons no primeiro encontro --- George disse. --- Você vai parecer um desesperado. E virgem.

--- Eu estou desesperado --- disse Jonas sorrindo.

--- \emph{E} virgem --- George observou. --- Por, isso tenha sempre consigo preservativos e lubrificante.

--- O quê? --- Jonas perguntou surpreso. --- Lubrificante?

--- No caso dela precisar --- George falou em um tom sério.

--- Nós só vamos sair\ldots\,--- Jonas começou a falar.

--- Se ela quiser, você não vai querer? --- George o interrompeu.

--- Mas ela não vai querer, eu acho.

--- Você coloca a mão no fogo por ela?

--- Eu\ldots\,Eu não sei --- Jonas disse.

--- Você é tão fraco, Jonas. A limerância está dominado você --- George disse\mudanca{,} cheirando próximo a Jonas. --- E passe um perfume.

Jonas cheirou a si mesmo. Ele voltou para sua casa de bicicleta, tomou banho, já que ele realmente precisava de um, e foi de carro até a casa de Falls. Ele chegou meia hora antes, então ficou pelo menos vinte minutos esperando dentro do carro antes de bater. Sim, ele passou perfume.

--- \emph{Hey} --- ela disse o abraçando e dando dois beijinho no rosto. Jonas ficou meio sem jeito\mudanca{,} já que nem mesmo a este ritual social tão tradicional ele estava acostumado. --- Você podia ter batido antes.

--- Você viu que eu cheguei?

--- Vi\mudanca{,} sim.

--- Oh\ldots\,--- ele ficou sem jeito --- Você está muito bonita --- Jonas disse tentando não tropeçar nas palavras. Havia ensaiado isso nos vinte minutos que passou esperando.

--- Obrigado --- Falls disse abrindo um sorriso.  --- E então, onde nós vamos?

--- O que você acha? O que for melhor para você.

--- Então vamos dançar --- ela disse empolgada.

Jonas engoliu em seco, ``Dançar?''

--- Dançar?

--- Você gosta? --- Falls perguntou.

--- Eu não sei dançar nenhum pouco. Espere\ldots\,--- Jonas disse\mudanca{,} abrindo a porta para ela, que entrou no carro e se sentou. \mudanca{Fechou} a porta.

--- Eu achava que você só tinha aberto a porta aquela vez para fazer uma média comigo --- Falls disse.

--- O cinto de segurança --- Jonas falou mostrando o próprio. --- Eu tenho essas manias, sabe? Eu tenho que fazer\ldots\,Então, onde fica esse lugar?

Falls teve que dar as instruções pelo menos três vezes e ainda avisar que ele já havia passado pela frente do lugar a pelo menos dois quarteirões. Ele ficou extremamente sem jeito, mas ela sempre parecia se divertir com esses erros. Um dos grandes medos de Jonas a respeito de relacionamentos não era apenas o medo da rejeição, o que dizer e outras coisas, e sim, o seu medo de errar. Isso o engessava. Ele evitava falar de seus gostos pessoais, como se divertia e tudo mais. Sua tática era fazer perguntas. Perguntas em cima de perguntas. Ele era um bom ouvinte, e afinal, qual o assunto predileto de uma pessoa? Ela mesma.

Para entrar no lugar havia uma pequena fila\mudanca{. Enquanto} esperavam, Jonas resolveu perguntar sobre algo que o incomodava um pouco\mudanca{. Desde} que aquela garota saíra do banheiro e \mudanca{se} encontrara com ele, pensava naquilo com uma espécie de temor que gostaria de esclarecer:

--- Vou te fazer uma pergunta \mudanca{que} talvez soe muito estranha\ldots

--- O quê? --- Falls perguntou\mudanca{,} desperta pela curiosidade.

--- Toda vez que eu vou na Novo Caminho eu vejo uma garotinha toda vestida de preto sentada lá na recepção. Ela\ldots\,Existe?

--- Bem\ldots\,--- Falls fez um rosto que misturava reflexão e preocupação. --- Sério?

--- Sério --- Jonas disse preocupado.

--- Não. Nunca vi. Eu até tive um arrepio agora --- ela disse lhe mostrando seu braço, do qual vinha um doce perfume. --- Por que sempre que fico sozinha ali, é como se houvesse alguém lá --- ela disse gesticulando.

Jonas engoliu em seco. Estava pálido e sério. Falls não aguentou muito tempo e começou a rir.

--- O quê? --- ele perguntou.

--- Você devia ver a sua cara --- Falls disse em meio ao riso. --- É a Sarah. É a filha de Satã.

Jonas riu menos preocupado.

\mudanca{O lugar} era bem menor que a Festa do Fim do Mundo, mas parecia estar tão lotado quanto. O que significava que a concentração de pessoas por metro quadrado era gigantescamente maior. Quando eles entraram naquele mar de pessoas, Jonas pensou, ``Agora eu vou saber como os átomos se sentiam antes do Big Bang''. Falls pegou em sua mão. ``Bom'', não conseguiu evitar a tempo que um sorriso escapasse para seu rosto.

--- Para não nos separarmos --- Falls disse levantando a mão deles.

``Nunca?'', Jonas pensou. ``Idiota''.

Ela o levou direto para o bar. Ela pediu alguma coisa, ele recusou.

--- Por que não? --- ela perguntou.

--- Eu não bebo --- Jonas disse.

A atendente voltou com um água mineral para ela.

--- Nem água? Como você sobrevive? --- Falls debochou.

Eles riram. Novamente Jonas se espremia entre costas, cabelos e robôs de 1984. O volume da música não estava tão alto quanto ele imaginava. Falls lhe disse que era porque ainda não havia começado ``pra valer''. Ela começou uma dança na frente dele que ficou lá parado, somente observando.

--- Então Jonas\ldots\,Diga-me algo sobre você --- ela disse\mudanca{,} em seu ouvido. Jonas prestou mais atenção na pequena distância a que estiveram, quando seus lábios quase que se tocaram.

Auto-apresentação.

O que dizer sobre si mesmo? Como responder a uma pergunta dessas? ``Quem sou eu?'' Ela\mudanca{, em seu} primeiro encontro\mudanca{,} já atirava nele um dos mais perigosos e intrincados enigmas existenciais de todos os tempos. Jonas nunca sabia o que dizer nessas situações. Como responder a uma pergunta da qual você não têm certeza alguma da resposta e ironicamente você seria a única pessoa que poderia dá-la? Não é nem mesmo uma pergunta de múltipla-escolha, o que deixaria um pouco da resposta ao acaso se você quisesse apostar em alguma delas.

Não poderia passar o resto da noite tentando chegar a uma resposta\mudanca{. Tinha} que dá-la ali\mudanca{,} naquele momento. Quando \mudanca{abriu} a boca para dizer qualquer coisa, um sujeito perto deles começou a vomitar. Jonas puxou Falls para o lado e saíram dali de perto.

--- Eu nunca vomitei --- Jonas disse para Falls.

--- Mentira --- Falls disse incrédula.

--- Sério. Por que eu mentiria sobre isso? --- falou no mesmo instante em que descobria que sim, já havia vomitado. Veio à sua mente memórias de sabores de vômitos desde a infância, até a última vez que fora exatamente na Novo Caminho logo após ter sido ressuscitado. Era sempre a mesma coisa. Ele ficava nervoso, sua boca soltava alguma coisa\mudanca{,} e seu cérebro logo pensava, ``Ei! Isso aí não se parece nenhum pouco com a verdade!''\mudanca{,} e então ele não podia fazer outra coisa a não ser continuar com a mentira.

--- Uau --- ela disse. --- Nunca achei que encontraria alguém que nunca vomitou em toda a sua vida.

--- Nunca bebi demais\ldots

--- Nem de menos --- Falls completou.

--- Nem de menos. Nunca comi nada estragado. Eu também não fiz um milhão de outras coisas --- ele acrescentou pensativo, tentando sair do centro do assunto. --- E você?

--- Oh, eu já vomitei algumas vezes --- Falls disse rindo.

A música começou a ficar bem mais alta nesse momento. O lugar começou a ser bombardeado por luzes \mudanca{de todos os tipos} que piscavam \mudanca{em vários} padrões e cores. Falls começou a dançar e motivá-lo a fazer o mesmo. Ele ficava sem jeito, olhava para ela e desejava poder se soltar como ela. Falls parecia feliz. Ele começou de forma tímida a acompanhar o ritmo da música. Primeiro imitando os movimentos que ela fazia. Como um sujeito de oitenta anos\mudanca{,} engessado, mas tentando.

Jonas teve poucas paixões. Nada especial o suficiente para ser digno de nota. Mas \mudanca{acreditava} no amor verdadeiro, uma fé cega, mas ainda assim uma fé. Chegou a sair uma ou duas vezes com algumas \mudanca{garotas}, mas sempre acompanhado de amigos mútuos que tentavam de alguma forma ser cupidos de seus colegas solteiros. Basicamente\mudanca{,} era juntar dois perdedores. Nunca funcionou. Ele se apaixonou algumas vezes. Até achava\mudanca{,} na época\mudanca{,} que era amor. Mas\mudanca{,} agora\mudanca{, já} não pensa assim. Sempre que ouve uma música triste ele pensa em cada uma delas. Talvez na próxima encarnação ele nasça bonito. Por isso sua visão da maioria dos relacionamentos era cínica e pessimista. Um valsa ao som da Balada de~O. Quem poderia culpá-lo?

Garoto encontra garota\mudanca{:} é onde começa. Há inúmeras variações\mudanca{,} como se é de se esperar. Garota encontra garoto, garoto esbarra em garota e a ajuda a pegar o material, garoto encontra garoto, garota encontra garota, e por aí vai. Aqui inicia-se a enfermidade e sua origem, embora possa ser reduzida também à biologia, é melhor compreendida através da imaginação. No momento em que duas pessoas se encontram e começam a conversar, tudo muito superficial, a imaginação de cada um deles começa a trabalhar a fim de preencher cada espaço em branco, cada lacuna. Em pouco tempo\mudanca{,} um olhará para o outro como se conhecessem desde sempre. Almas gêmeas é um termo muito utilizado nesta fase. Em um ``clique''\mudanca{,} um se tornou\mudanca{,} aos olhos do outro\mudanca{,} o príncipe ou a princesa, a razão de viver\mudanca{,} e aquela merda toda.

Nada mais natural e errado \mudanca{do} que iniciar um relacionamento. O cérebro não está tomando as melhores decisões\mudanca{,} pois ainda sofre os efeitos do sistema de recompensa que age nestes casos de paixão de maneira parecida com a cocaína\mudanca{. Não} conhece em absoluto o outro, mas é como se conhecessem. Ou pelo menos querem achar que sim. Se nós temos dificuldade em dizer quem nós somos, o que poderemos dizer dos outros? Eles passam a se amar incondicionalmente, o que quer que isso signifique. Então \mudanca{estão} seguros de que é certo dar o próximo passo e juntam as escovas de dentes. Se casam. É onde está o trágico erro dos amantes.

\mudanca{``Por que é um erro?''}, você pode se pergunta, \mudanca{``se estão ambos iludidos e retirando prazer, companhia e cumplicidade?''} Porque é agora que vão se conhecer, mas se conhecer de verdade. E quando isso acontece\mudanca{,} a pessoa não interpreta desta forma\mudanca{:} ela interpreta como se o outro estivesse agindo de forma imprevisível, de forma errada\mudanca{;} porém\mudanca{,} é a forma do outro agir \mudanca{desde sempre}. Muitos relacionamentos sobrevivem a esta fase e mais tarde se tornam casamentos falidos, que poderão engordar as estatísticas de traição ou divórcio\mudanca{,} e do qual a felicidade se esquivará.

Mesmo os que sobrevivem acabam passando por esta que é a pior fase: o estranhamento. Um dia\mudanca{,} ``clique'', tudo sumiu. Assim como no primeiro dia, \mudanca{em} um passe de mágica, o outro era o ser mais especial no mundo. Em outro, assim como veio, descobre que nunca conheceu em absoluto a outra pessoa\mudanca{, que é uma completa estranha}. Esta fase é facilmente reconhecível no rosto de casados \mudanca{há} um ou dois anos\mudanca{,} e em homens e mulheres falando de ex-namorados, pessoas com que juravam eternidade e agora falam como se fossem estrangeiros.

Nesse processo do total conhecimento ao total desconhecimento\mudanca{,} colhemos alegrias\mudanca{,} mas nos ferimos muito e\mudanca{,} mesmo feridos\mudanca{,} entramos em outras e outras danças, como se de alguma forma quiséssemos nos machucar, como se esse fosse o motivo de se relacionar. Masoquistas sentimentais, bailando como~O.

Mas será que era assim, exatamente como ele concebia? Ele tocava a mão de Falls e sentia todo o seu corpo \mudanca{vibrar em resposta}. Como se ela tivesse acordado uma parte dele que era familiar, mas nunca havia se revelado totalmente. Então\mudanca{,} vendo todas aquelas pessoas, no auge de suas vidas, na epítome de sua geração, Jonas pensou na Dança Macabra.

É uma alegoria do final do período medieval sobre a universalidade da morte: não importa o estatuto de uma pessoa em vida, a dança da morte une a todos. \emph{La~Danse Macabre} consiste na representação de uma~Morte personificada\mudanca{,} conduzindo um fileira de figuras de todos os estratos sociais\mudanca{,} dançando em direção aos seus túmulos --- tipicamente com um imperador, rei, papa, monge, adolescente e bela mulher, todos numa forma esqueletal. Estas representações foram produzidas sob o impacto da Peste Negra, que lembrou as pessoas de quão frágeis eram suas vidas e quão vãs eram as glórias da vida terrena. Quem não sabe que vai morrer? Por isso temos que aproveitar cada minuto de nossas vidas. Jonas sentiu em si esse desejo de aproveitar. Falls dançava junto a ele. Ele se sentia ao mesmo tempo orgulhoso e confiante.

Esta é a sua vida. Cada hora a mais é na verdade, uma hora a menos.

--- QUER SABER DE UMA COISA? --- Falls falou bem pertinho de seu ouvido, tendo que gritar para ele ouvi-la. Os lábios dos dois ficaram a menos de dois centímetros um do outro.

--- O QUÊ? --- ele perguntou.

--- NÃO HÁ JEITO CERTO OU JEITO ERRADO DE DANÇAR. FAÇA OS SEU PRÓPRIOS PASSOS. SE DIVIRTA! --- ela disse.

DANCE DANCE DANCE.

Deve dançar. Enquanto a música estiver tocando, você deve continuar a dançar. Entende o que quero dizer? DANÇAR, CONTINUAR DANÇANDO. Não deve penar no motivo e nem no sentido disso, pois eles praticamente não existem. Se ficar pensando nessas coisas, seu pé ficará imóvel. Uma vez parado, já não será capaz de agir. Por isso, não deve parar de mover os pés. Por mais que lhe pareça uma tolice, não deve ligar. Deve continuar dançando, dando os passos. Deve ir amolecendo, mesmo que aos poucos, tudo o que estava completamente rígido. Use tudo o que pude usar. Dê o máximo de si. Não há o que temer. Só lhe resta dançar. E dançar de modo exemplar. A ponto de todos ficarem admirados. Pois, assim, talvez eu consiga ajudá-lo. Por isso, dance enquanto a música estiver tocando. DANCE ENQUANTO A MÚSICA ESTIVER TOCANDO!!!

É agora. Falls o olhou nos olhos enquanto se balançava, desfazendo aquele penteado tão bonito. Então Jonas dançou. Com a desenvoltura de um pato, mas realmente se esforçando. ``FAÇA OS SEU PRÓPRIOS PASSOS. SE DIVIRTA!'' Não havia por que achar que todos seguiriam os passos da Balada de~O. Não há qualquer razão para acreditar nisso. Ele poderia fazer seus próprios passos. Se divertir.

--- É ISSO MESMO! --- Falls disse\mudanca{,} quando o viu imerso no espírito da coisa. Nem parecia ser ele mesmo. Então no meio da empolgação, Falls entrelaçou seus dedos na nuca de Jonas. Foi como se eles estivessem sós naquele lugar e tudo o mais fosse um cenário borrado, caótico e indiscernível. Um isso-daquilo, uma miscelânea caleidoscópica de Kaspar Hausers. O tempo parou. A mágica aconteceu.

``O melhor beijo de toda a minha vida'', Jonas pensou.

Quando acordou\mudanca{,} no dia seguinte\mudanca{,} estava todo jogado na cama. ``Droga! Foi um maldito sonho!'', pensou enquanto abria os olhos. ``Ainda que o melhor sonho de todos''. Pixel, como todos os dias\mudanca{,} veio até ele andando, pedir para encher seu prato. Jonas foi se levantar e simplesmente não conseguiu. Toda a musculatura de sua perna estava dolorida. Ele pegou o celular para ver que horas eram. Quase meio-dia do sábado. Uma mensagem de texto. ``George?'', ele pensou. Não. Era de Falls\mudanca{,} agradecendo pela noite que tiveram. Jonas deitou novamente. Não foi um sonho, foi melhor que isso. Ele estava feliz.
