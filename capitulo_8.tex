\chapter{Os Renascidos}

Jonas havia resolvido ir uma das reuniões dos Renascidos para ver como era. Aquilo tudo o amedrontava um pouco, mas igualmente o fascinava. Sentou-se em um ponto intermediário entre o fim e o início das cadeiras, exatamente como sempre fez na escola. Logo, um homem de certa idade se pôs à frente de todos em um pequeno palco. Era impossível não notar o seu olho roxo. Jonas ficou imaginando por onde um senhor dessa idade teria andado para conseguir o olho roxo? Em algum clube da luta?

--- Boa noite a todos --- o homem disse recebendo uma reposta coletiva, ainda que tímida. Ele mexeu em alguns papéis e colocou um óculos, e então leu em voz alta para a plateia:

--- Esta é a hora da morte e renascimento --- dizia em tom solene. --- Aproveita esta morte temporal para atingir o perfeito estado. Ilumina-te. Concentrado na unidade de todos os seres vivos. Mantido sobre a Luz Clara. Usa-o para alcançar o entendimento e o amor. Quero antes de mais nada, falar sobre os nossos encontros. A primeira hora trata-se de uma palestra dirigida em sua maior parte para candidatos ao procedimento. Após esta uma hora temos o compartilhamento de experiências. Todos são convidados a participar dos dois processos. O modelo que seguimos aqui é de uma desconferência. A desconferência não tem agenda definida, embora a dividamos em duas fases com objetivos distintos. Ela possui quatro regras --- ele disse mostrando três dedos levantados. Jonas automaticamente riu\mudanca{,} atraindo várias olhares para ele, o \mudanca{que o deixou} totalmente sem \mudanca{graça e o fez afundar} na cadeira. --- E uma lei. As quatro regras são: um, seja quem for que veio, é a pessoa certa. Dois, o que quer que aconteça, é apenas aquilo que deveria ter acontecido. Três, quando quer que comece é a hora certa e por último, é como fritar bacon, quando você acha que acabou, então acabou. E sem esquecer a lei dos dois pés\mudanca{,} que diz que se a qualquer momento você se encontrar em uma situação em que não esteja aprendendo ou contribuindo, use seus dois pés e dirija-se a um lugar mais do seu gosto. A porta está sempre aberta. O banheiro é a primeira porta à esquerda. E sim, ele pode ser usado. Apenas lembrem-se de dar descarga. Eu me chamo Edgar, se for do interesse de alguém --- falou após se lembrar de não ter dito o nome, com um sorriso simpático no rosto, guardando o papel que segurava.

--- Morte não existe, --- ele disse com um sorriso nos lábios. --- e é por isso que estamos aqui esta noite. Tudo o que vive, vive para sempre. Somente o invólucro, o que é perecível, desaparece. O espírito não tem fim, é eterno, não tem fim. A Novo Caminho é uma clínica especializada em induzir experiências de quase-morte em seus pacientes. Temos registros de experiências de quase-morte desde a antiguidade. Com o desenvolvimento da ciência moderna, novas drogas e equipamentos nos permitem reverter quadros que antes seriam mortais\mudanca{; logo,} aumentamos a incidência de pessoas que morreram por alguns minutos e \mudanca{voltaram à} vida. Nenhuma das experiências é totalmente igual. Cada pessoa passa por algo diferente, ainda que tenha traços ou temas em comum. Mas há algo, talvez o que nós buscamos oferecendo nossos serviços, que une todas \textsc{eqm}s: as pessoas que passam por elas mudam radicalmente a forma como agem em \mudanca{suas vidas}. Mais e mais \textsc{eqm}s passaram a acontecer. O que podemos dizer sobre elas? Ao trazer uma pessoa da morte, alguns médicos salvavam a vida dessas pessoas não apenas uma, mas duas vezes. Física e espiritualmente.

--- O \emph{Método Ars Moriendi} --- continuou Edgar --- permite que cada pessoa que voluntariamente queira participar fique em um estado tal que possa ser \mudanca{declarada clinicamente morta,} para então ser ressuscitada. Essas quatro pessoas aqui --- ele disse apontando para aqueles sentados em cadeiras em cima do palco --- irão esta noite compartilhar um pouco de suas impressões ao saltarem. Aliás, nós nos referimos ao procedimento de \mudanca{``salto''} em homenagem a Thomas Hobbes. Agora quem irá falar é Arthur, o homem que detém o recorde de nove saltos.

As pessoas começaram a murmurar algo para outra. O espanto foi geral. Afinal, muitos estão hesitantes em realizar um salto e o sujeito à frente deles já fez isto nove vezes. Talvez não fosse tão perigoso assim. Ou talvez fosse \emph{exatamente} isso que o pessoal da Novo Caminho queria que eles pensassem.

--- Boa noite --- disse Arthur para a plateia. --- Muitas pessoas se perguntam e com certeza irão perguntar\mudanca{,} a cada um que saltar, como é o outro lado. Eu sempre digo: eu vi o outro lado, e ele é lindo. Muitas vezes\mudanca{,} eles ficam curiosos para saber se aquilo que você viu confirma ou não suas crenças. É natural. Se confirma\mudanca{,} talvez eles olhem para você e sintam-se bem. Se você disser algo que contrarie\mudanca{,} o olharão de cabo a rabo e perguntarão quem você pensa que é. Que é um louco, mentiroso. É assim que as pessoas são. Mas\mudanca{,} se vocês também estão se fazendo essa mesma pergunta, a resposta pode ser um tanto quanto estranha. Ao que parece, as pessoas podem escolher a forma com preferem ser salvas. As crenças pessoais de uma pessoa sempre colorem suas experiências. Em outras palavras, só vai para o Inferno aqueles que realmente acreditam nele --- concluiu\mudanca{,} um pouco hesitante.

Ele arrancou risadas da plateia. Jonas olhava \mudanca{com inveja e admiração} para eles. Eles tinham visto o que ele ansiava em ver, mas tinha medo de fazer.

--- Boa noite --- disse outro dos homens se dirigindo ao púlpito. --- Meu nome é Frederico e faço parte de nosso pequeno grupo informal de estudos de \textsc{eqm}s. Nós não temos um nome propriamente dito, mas gosto de nos chamar de Renascidos. Nem todas as pessoas que se submetem ao \emph{Método Ars Moriendi} conseguem ter uma \mudanca{experiência. Para essas,} é como se acabassem de acordar da maior ressaca de suas vidas. Há, ainda que muito baixa, uma taxa de mortalidade. Mais seguro\mudanca{,} no entanto, que viajar de automóvel. E\mudanca{,} por fim, nem todas as \textsc{eqm}s são agradáveis como a grande maioria. Algumas pessoas passam por experiências terríveis e assustadoras. A ocorrência das mesmas está entre quatro~e seis por~cento. É muito importante que tenham todas as informações na hora de pesar e avaliar se devem ou não realizar o salto. Cada \textsc{eqm} ocorre em \mudanca{cinco~etapas}.

--- A primeira delas é a paz. No início da experiência, a dor desaparece. Noções de espaço e tempo cessam. Sessenta por cento das pessoas passam por essa --- ele disse mostrando seis dedos. Jonas se revirava na cadeira de plástico\mudanca{,} se perguntava se essas palestras eram realmente necessárias. Será que todos usavam\mudanca{, mesmo,} essas informações na hora de decidir\mudanca{? Ou}simplesmente decidiriam pelo que estivessem mais inclinados a fazer?

--- O segundo passo é a viagem --- ele \mudanca{continuou}. --- Você se separa de seu corpo físico e passa por uma experiência fora-do-corpo. Alguns flutuam sob seu corpo e podem até mesmo dar detalhes do que se passa com \mudanca{ele}, o qual veem de cima. Essa embora não ocorra com a maioria\mudanca{,} é realmente muito útil para a ciência. Pois\mudanca{,} em cada sala, colocamos geradores de números aleatórios ou objetos fora de contexto, sem conhecimento dos pacientes, claro, para que casos relatados possam oferecer dados empíricos sobre a validade de uma \textsc{eqm} como um tipo mais elevado de experiência\mudanca{,} e não apenas uma alucinação ou uma resposta do cérebro à falta de oxigênio. Nesta categoria de viagens\mudanca{,} incluem-se muitos tipos, até mesmo viagens espaciais como Carl Gustav Jung relatou. A terceira fase é o túnel. É uma etapa transitória de escuridão. Algumas vezes é avistada uma luz no fim do túnel, e \mudanca{os pacientes} vão nessa direção. Entrou para o imaginário popular como o paradigma de uma \textsc{eqm}s. A própria luz ou o que está no fim do túnel é o nosso quarto item.

--- Algumas pessoas --- ele disse se virando para alguém dos que se sentavam nas cadeiras --- dizem que se trata da luz mais clara do Universo e no entanto não ofusca a visão. O quinto passo é a fronteira. O que existe além disso parece ser colorido com a crença de cada um, mas sabem que é este ponto em que a morte é irreversível. Acontece também em \textsc{eqm}s além desses cinco fases clássicas, a retrospectiva, o famoso ``vi a vida passar diante dos meus olhos''. A escolta de parentes mortos ou companhia de algum ente querido. Um ser iluminado que o guia ou o acolhe. A identidade deste também é extremamente dependente da fé. E um dos mais cobiçados eu acho que é o conhecimento global, quando ocorre o que os místicos chamam de morte do ego.

Morte do ego. Isso intrigou Jonas.``O que isso afinal queria dizer?'' De qualquer forma, logo outro deles começou a falar atropelando as reflexões dele.

--- Enquanto os anteriores falaram da experiência em si, afim de informá-los sobre aspectos gerais do que se é esperado da Novo Caminho, chegou a vez de falar da própria. Sou um dos médicos responsáveis, Dr.~Patrick Kafka. Foi aqui\mudanca{,} em nossa cidade\mudanca{,} que a Novo Caminho iniciou seu trabalho há alguns \mudanca{anos,} graças ao método desenvolvido pelo meu cunhado\mudanca{,} Dr.~Roberto Mouir. Hoje em dia\mudanca{,} já contamos com uma série de franquias \mudanca{em} nosso país\mudanca{,} e já estão sendo inauguradas unidades na Inglaterra, Estados Unidos, Japão, França e Alemanha. O objetivo principal não é apenas permitir que uma pessoa tenha uma experiência de quase-morte. Estamos oferendo uma chance real de mudar sua vida, imbuí-la de valor e propósito. Efeitos \mudanca{observados} na maioria das pessoas que passaram por essas experiências. Nosso ramo é \mudanca{o de} mudar a vida das pessoas. E estamos falando de mudanças reais e profundas. Quem não sabe que vai morrer? Mesmo assim\mudanca{,} dificilmente alguém, por mais inteligente que seja, consegue aplicar de forma real e objetiva uma filosofia como a pregada pelo \emph{memento mori} ou \emph{carpe diem}. Nós nos apegamos a vida. Achamos que vamos tê-la indefinidamente. Somente quando você perde tudo é que está pronto para a mudança. É isso que o \emph{Método Ars Moriendi} propõe. E funciona. É um procedimento não cirúrgico, mas que causa por alguns segundos morte clínica no paciente que, devido a uma série de fatores, nem sempre pode ser revertida. Ele é voluntário e cada um de vocês deve assinar um contrato em que exime judicialmente a Novo Caminho de qualquer responsabilidade. Sua família será notificada e receberá um montante pelo seguro. Mas\mudanca{,} em nenhum cenário\mudanca{,} eles poderão culpar qualquer indivíduo desta empresa como responsável pela sua morte. Existem três pacotes. O primeiro deles e mais barato é o Plano~Bronze. Você passará pelo \emph{Ars Moriendi}, receberá acompanhamento por algumas horas após o procedimento nas instalações da Novo Caminho e será liberado. O Plano~Prata é o intermediário\mudanca{. Além} do descrito no anterior\mudanca{,} você também \mudanca{receberá acompanhamento} psicológico por Edgar --- ela disse apontando para o sujeito de olho roxo. --- O Plano~Ouro acrescenta ao Plano~Prata uma série de outros benefícios que alguns podem querer optar. Valores, assim como uma descrição mais detalhada de cada um dos \mudanca{planos,} pode ser encontrada nos folhetos que \mudanca{distribuiremos}.

Jonas pegou um dos folhetos e seus olhos procuraram imediatamente a parte onde se mostrava os preços. \mudanca{Um} pouco altos, mas nada realmente tão caro que ele não pudesse pagar com um ou dois meses de trabalho. Pelo menos até o Plano Prata, que seria o que escolheria.

--- Perguntas? --- ele disse para a plateia.

--- Qualquer um dos pacotes tem o direito de escolher qualquer horário para realizar o salto? --- um homem perguntou ao mesmo tempo em que lia a informação no folheto.

--- Sim. A Novo Caminho deixa a pessoa escolher o horário que ela se sinta mais confortável. Há desde pessoas que querem determinados horários por causa de algum significado especial ou outras que somente podem realizá-lo em determinadas horas.

``Pelo menos podemos escolher a hora de nossa morte'', Jonas pensou\mudanca{,} sorrindo com o canto da boca.

--- Vocês também têm direito a uma canção --- Dr.~Patrick falou\mudanca{,} sorrindo.

Um homem levantou a mão.

--- Olha, eu estou pensando em colocar a vida nas mãos de vocês. Acho que o mínimo que podem fazer é dizer o que diabos esse \emph{Método Ars Moriendi} faz.

O Dr.~Patrick ficou um pouco transtornado com a pergunta, mas logo surgiu um sorriso.

--- Ele possui cinco fases distintas --- ele começou. --- A primeira, é a administração de \emph{tiopentato de sódio} que é um sedativo. Administramos de duas a cinco gramas.

--- Tiopental? --- interrompeu o homem.

--- Exato.

--- Usam isso como soro da verdade, não é mesmo?

--- Sim. Sim --- respondeu o doutor claramente perturbado por ter sido interrompido --- Com tal dose, o paciente é nocauteado em dez segundos. A segunda fase é a administração de \emph{bromuro pancurônio}, cem miligramas. É um poderoso relaxante muscular que entre quinze e trinta segundos terá paralisado o diafragma e o pulmão. Por fim, \emph{cloreto de potássio} faz o coração parar de bater. A quarta fase é a de ressuscitação com o desfibrilador.

--- Mas há algo errado com isso, não?--- o homem perguntou. --- Eu sou fã de um seriado médico, \emph{Grey's anatomy}, eu não sou médico nem nada. Quando o paciente fica em linha plana, a simples desfibrilação não vai ter efeito algum. Somente podemos trazer um paciente de assístole, o termo técnico para isso, quando se faz massagem cardíaca no próprio coração, com o peito do paciente aberto. Se fosse tão fácil assim ressuscitar alguém, morrer seria fácil como se dar um passeio. Muitos filmes e seriados abusam do desfibrilador como se ele fosse um ressuscitador. Não é bem assim.

--- Até há pouco tempo era essa a verdade. Muitos poucos sobreviviam à assístole. Não havia mais atividade elétrica, pois ele se despolarizava. Mas o cenário mudou, atualmente. A soma de um novo e mais eficiente desfibrilador e uma nova droga experimental\mudanca{,} que eles guardam a composição à sete chaves. Essa é a quinta fase do processo. Administrar direto no coração essa droga. Chama-se \emph{Lazarus}. Foi sintetizado a partir de uma planta descoberta na Amazônia chamada \emph{Chamalla}. O laboratório faz muito mistério acerca da fórmula química do mesmo, sabemos apenas que é a partir da \emph{Chamalla} e a adição do ``elemento~\textsc{x}'', cuja composição é claramente um segredo industrial. Possui um efeito virtualmente imediato. Vai revolucionar a medicina e permite a existência do \emph{Método Ars Moriendi}. Ainda terá sua dose de risco. Morrer é como dar um passeio de asa delta. Perguntas?

Ao sair da reunião e se dirigir para fora, Falls estava em pé segurando um livro enorme.

--- Olá --- Falls lhe disse sem ânimo e encostada na parede, morrendo de tédio.

--- Fazendo hora extra? --- Jonas disse\mudanca{,} depois de cumprimentá-la com um aceno.

--- Na verdade esperando minha carona, que talvez vá atrasar muito.

--- Você parece cansada --- Jonas notou.

--- Amanhã eu tenho prova e queria dormir cedo --- ela disse bocejando.

--- Legal. O que você faz? --- Jonas perguntou.

--- Psicologia. Eu não acho que seja o sonho de muitas garotas serem recepcionista. Bem, paga as contas --- ela disse como querendo se desculpar pelo que fazia.

--- Interessante. Quer que eu te dê uma carona? Eu estou de carro e posso te levar --- disse Jonas nem mesmo acreditando que acabara de dizer aquilo.

--- Sério? --- perguntou Falls passando a mão no cabelo.

--- Claro --- ele disse, já arrependido de ter feito o convite.

--- Você segura? --- perguntou Falls sem esperar a resposta e já passando livro enorme para Jonas segurar.

Ela pegou um post-it, escreveu alguma coisa e pregou na parede ao lado do balcão.

--- Vamos? --- ela disse. --- Espero que você não tente nada. Eu estou com meu spray de pimenta bem aqui.

Eles riram.

--- Quem era aquele sujeito?

--- Tomás de Torquemada.

--- Que nome diferente --- falou Jonas.

--- Assim como Falls --- ela disse. --- É francês.

--- Seus pais são franceses?

--- Não --- ela disse meio irritada. --- Mas \emph{todo mundo} me pergunta isso. É só um nome que \emph{eles} acharam bonito. Falls das Neves. Desculpe, pelo tom de voz. É que todo mundo me faz essa pergunta.

--- Tudo bem. Na verdade eu que tenho que me desculpar. Eu não queria ser como todo mundo.

--- Tudo bem --- ela disse.

--- Das Neves? --- perguntou Jonas olhando para ela\mudanca{,} que sorria.``Covinhas'', ele pensou.

--- É --- disse ela.--- Exatamente como o Abominável Homem das Neves, se é isso que está pensando.

--- Abominável Falls das Neves --- Jonas falou fazendo uma voz cômica.

--- Você me acha abominável --- ela afirmou em um tom neutro.

--- Não\ldots\,Claro que não\ldots\,Pelo contrário --- ele disse sem jeito, logo parando de andar. --- Esse é o meu carro --- ele disse parando ao lado de um dos vários carros estacionados e apertando um botão em sua chave, desligando o alarme. Abriu a porta para ela.

--- Então você é um cavalheiro, Jonas --- Falls disse\mudanca{,} com satisfação.

Ele entrou no carro, e se sentou. Tremia um pouco. Não queria ficar olhando toda hora para ela com medo tanto de parecer uma espécie de maníaco sexual ou perder o controle do carro. Mas\mudanca{, mesmo assim,} olhou. E foi pego. Ela abriu um sorriso que o deixou confuso e assustado.

--- O cinto --- ele disse apontando para o mesmo.

--- Então Jonas\ldots\,--- ela disse colocando o cinto --- sua namorada não vai ficar brava de você me dar uma carona?

--- Eu não tenho uma namorada --- ele disse\mudanca{,} manobrando para sair com o carro da vaga em que estacionara.

--- Sua esposa? --- ela perguntou, sem jeito.

--- Também não sou casado --- ele disse\mudanca{,} achando graça na situação.

--- Seu\ldots\,Namorado? --- ela perguntou sem jeito\mudanca{,} e fazendo uma careta.

--- Não. Eu não sou gay também --- ele disse se sentindo contra a parede. Não havia como fugir.

--- Solteiro --- ela disse em um tom que parecia tanto ser uma afirmação quanto uma pergunta, era impossível discernir.

--- E você\ldots? --- Jonas perguntou\mudanca{,} meio sem jeito. % Dar uma olhada nessa sucessão infinita de “meio sem jeito”!

--- Eu? --- Falls disse. --- Não. Não tenho namorado\mudanca{, nem} sou casada.

Ela disse isso e ficou olhando pela janela. Então, como se tivesse lembrado de algo importante adicionou:

--- E nem mesmo sou lésbica --- ela disse.

--- Eu já tinha começado a ficar preocupado --- Jonas rindo.

Ela lhe disse qual casa era. Ele parou o carro e pôs em ponto morto. \mudanca{Jonas} ficou sem saber o que fazer, então\mudanca{,} tirando o cinto, disse:

--- Espere aí. Saiu do carro, deu a volta e foi abrir a porta para ela.

--- Assim você vai me deixar mal acostumada --- ela disse.

--- Que nada --- Jonas disse\mudanca{,} tendo certeza de que ela mal se lembraria dele pela manhã.

Eles começaram a andar em direção a casa dela. Em ambos os rostos era nítido que eles não tinham ideia nenhuma do que fazer ou dizer em seguida. Ela parou em frente à porta e começou a procurar por suas chaves dentro da bolsa.

--- Eu sempre encontro tudo, menos o que eu procuro --- ela disse\mudanca{,} olhando para ele e vasculhando o conteúdo da bolsa com uma mão.

--- Achei! --- ela disse\mudanca{,} enquanto sua mão emergia com um molho de chaves.

``É, agora que eu chamo ela para sair qualquer dia desses ou o telefone dela'', pensou Jonas, que motivando-se, disse:

--- Então\ldots\,Você está entregue --- Jonas disse sem jeito. --- Boa noite.

--- Boa noite --- ela respondeu meio sem jeito, olhando para ele e não sabendo o que fazer. Ela então só acenou, entrou e fechou a porta.

Jonas ficou ainda um tempo lá fora se amaldiçoando por ser um completo idiota que não sabe agir ou tomar a iniciativa.``Se eu pedisse o telefone dela, provavelmente nunca teria coragem de ligar mesmo'', disse ele, de alguma forma se consolando.

``Então é isso'', Jonas pensava\mudanca{,} enquanto voltava para casa, ``eu irei saltar''. Ele resolveu ligar assim que ele \mudanca{encontrasse} coragem para tanto. Era melhor ligar o ``foda-se'', ignorar os riscos e se jogar sem buscas irritantes por fato \& razão. Ele preferia arriscar tudo a ter uma vida inteira com medo por causa dessa grande incerteza. Mas só de pensar que ele teria que ligar na manhã seguinte para a Novo Caminho e falar com Falls, já o fazia tremer por antecipação.
