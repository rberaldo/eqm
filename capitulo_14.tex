\chapter{Transatlanticismo}

``Eu sou apenas um personagem'', pensou Jonas. ``Mas quem não é?''. Na maior parte do tempo nós estamos simplesmente tentando corresponder às expectativas dos outros. É através de um personagem que somos vistos e percebidos. Nós jamais chegamos a nos conhecer. Nunca podemos ter uma ideia clara de como vamos reagir em determinadas situações. Quem já ouviu a própria voz gravada, sabe o quão errada \mudanca{é a impressão que temos dela}. Assim também é com o eu. É através de personagens que nosso eu interage com o mundo, \mudanca{com} as pessoas. ``Eu'' penso, logo existo. Mesmo quando falamos ``eu'', se tem a sensação de que se fala sobre outra pessoa. Em algumas delas somos personagens simpáticos, o mocinho da história, em outras podemos fazer o papel de antagonistas, interesses românticos ou nada mais do que ser um figurante. Nossos personagens revelam aspectos da pessoa, alguns belos, outros terríveis. Quem diz que conhece alguém totalmente, cai em erro. Todas as pessoas estão fadadas a eternamente estarem em lados opostos, separados por um oceano de dúvidas, mal entendidos e interpretações erradas. Transatlanticismo.

--- O que você está imprimindo? --- perguntou George para Jonas\mudanca{,} no último dia do Natal.

--- O \mudanca{\emph{Bardo}}. Vou levar para ler na casa dos meus avós.

--- Eu sempre preferi paladinos.

--- É o \mudanca{\emph{Bardo Thödol}}.

--- O que é isso? Uma classe de prestígio?

--- Não\ldots\,--- Jonas disse\mudanca{,} rindo. --- É o livro tibetano dos mortos. É uma espécie de guia para o mundo dos mortos. Como meu avô está com um pé na cova, talvez seja interessante ele ler.

--- Você vai levar isso para ele?

--- Lógico que não, George. Eu e meu avô praticamente nunca nos falamos. Eu não sou exatamente o neto predileto.

--- Eu sempre passo o Natal com meu tio --- George falou. Jonas já sabia disso. Todo começo de ano, George gastava um tempão contando alguma de suas aventuras com seu tio. O pai de George morreu quando ele era bem pequeno. Ainda morava na casa dos avós, junto com a mãe e o tio. Ele gostava de dizer que era o Robin de seu tio. Jonas sempre lembrava das possíveis implicações sexuais disso, \mudanca{advertia George a} não dizer isso muito alto, mas ele continuava mesmo assim. Se despediram como faziam todos os dias, apenas com um aceno de cabeça, só que dessa vez apenas se veriam daqui a uma semana.

As crianças adoram dezembro. Jonas detestava. Sentia-se melancólico e com uma vontade imensa de ficar sozinho com seus pensamentos. Mas ele nunca conseguia, pois sempre estava rodeado de parentes\mudanca{,} na casa dos avós\mudanca{,} para reviver as velhas mágoas, ficarem bêbados e terem discussões metalinguísticas sem nem mesmo saber o que diabos significa metalinguagem. É sempre a mesma coisa: todos os planos e assuntos e aspirações referem-se à própria ocasião, o que está acontecendo ou o que vai acontecer em seguida.

--- Resolvi fazer aqui na sala, que é mais arejado. Tá fazendo um calor esses dias, né? --- dizia sua tia Beth para os outros.

--- Gostou do patezinho de atum? É receita nova. A Lenorá que passou, aquela minha amiga do salão --- perguntava outra tia a algum primo dele.

--- Deixa só chegar o Ruy com a namorada dele que a gente começa a brincadeira de amigo secreto. Vai ser assim\ldots\,--- começava um tio seu explicando passo-a-passo tudo o que ele fazia desde que era pequeno.

--- Ô\mudanca{,} tia, o Lucão tá sentado na sua cadeira de sempre, a senhora vai deixar barato? --- desafiava um sobrinho.

--- Mãe\ldots\,Chega de martíni. Lembra da diabetes! Você não tem jeito mesmo! --- falava a única prima da mesma idade de Jonas, \mudanca{que ignorava completamente sua existência}.

--- Hum\ldots\,O pavê tá ótimo! Quem fez? --- perguntava alguém\mudanca{,} assaltando a geladeira. Jonas olhou para todos na sala e começou a contar. Se ele estivesse certo não demoraria nem dez segundos até alguém responder. Um, dois, três, quatro\ldots

--- É pavê ou pá comê? --- Bingo! Agora todos riem.

--- Eu vou estourar os meus miolos --- Jonas falou para um de seus primos menores, com cinco anos\mudanca{,} no máximo.

--- Eu fiz cocô --- respondeu o garoto\mudanca{,} naquela que deve ter sido a sua conversa mais sincera nos dias que esteve com eles.

Quanto mais olhava para seus familiares, \mudanca{mais} se perguntava quem ele era e quem eram eles. ``Somos um bando de desconhecidos que compartilhamos parte do \textsc{dna}, o mesmo sobrenome, mas que tem empregos, gostos, orientação política e anseios diferentes. Nos reunimos todo ano por um par de dias, pra quê?''\mudanca{,} Jonas refletia.

--- Viu o jogo do Coringão? --- seu tio lhe perguntou\mudanca{,} trazendo\mudanca{-o} de volta para aquilo que chamamos ``realidade''.

--- Eu não acompanho futebol --- Jonas respondeu. Não ligar para futebol no Brasil e dizer isso equivale a se declarar ateu no Vaticano.

--- Gosto é que nem braço --- seu tio disse\mudanca{,} antes de tomar um gole de cerveja Duff. --- Têm gente que \emph{não} tem.

Jonas foi se sentar na varanda da casa de sua avó enquanto todos os outros falavam de suas vidas. Interrompiam uns aos outros para emendar assuntos que mudavam completamente o rumo da conversa apenas para se vangloriar de coisas que não tinham importância nenhuma ou eram, para citar George, irrelevantes. A valorização da mediocridade.

--- E aí, Jonas --- falou um de seus primos com um tom de empolgação que contrastava com o estado de espírito de Jonas tanto quanto colocar lado a lado um Picasso e uma obra de H.G.~Giger.

--- Só pensando --- Jonas disse.

--- Quer uma cerva?

--- Não, obrigado.

--- Você lembra muito do seu pai?

``Obrigado por perguntar''.

--- Claro. Não dá para simplesmente esquecer.

``Ainda mais com todo mundo \mudanca{te} lembrando \mudanca{d}isso a todo maldito instante\mudanca{,} mesmo já tendo se passado anos''.

Mais um dia e ele voltaria para sua casa. Por que ele continuava insistindo em vir nessas reuniões de família? No caminho para a casa de seu tio ele jurou nunca mais voltar, e dessa vez era para valer.

--- Eu reformei o carro --- seu tio lhe disse quando ele ficou parado em frente à garagem. Ele estava com uma pequena garrafa de cerveja na mão e tomou um gole. --- Se você quiser, pode levá-lo.

--- Tudo bem, não precisa --- Jonas disse com o olhar perdido.--- Esse carro me trás más lembranças.

O tio de Jonas deu uma batidinha em suas costas. Jonas odeia isso.

--- Quer uma, sobrinho? --- ele disse mostrando a garrafa em sua mão.

--- Eu não bebo --- Jonas disse pela milésima vez a alguém de sua família. Isso também lhe trazia más lembranças.

Na véspera do Natal seu avô ficou muito doente. Toda a família foi para o hospital. Um lugar para onde só vamos dizer adeus. Na sala de espera, o tédio associado à espera das más notícias fazia com que todos parecessem como se estivessem com dor de barriga. A televisão em um volume inaudível para ouvidos humanos entretinha a si mesma.

Então veio repentinamente a Jonas que todos nossos planos são pequenas preces para o Tempo. O médico que o atendeu havia dito a um dos tios de Jonas que em breve atualizaria a família sobre o estado de seu avô. Quando ele apareceu na porta da sala, todos se levantaram. Ele, suas tias, sua mãe e todos os primos.

--- Nós fizemos o possível \ldots\,--- disse o médico tirando os óculos. Uma de suas tias começou a chorar compulsivamente. E todos se olharem sem saber o que dizer. --- E tudo ficará bem. Só foi um susto.

--- O quê? --- um dos tios de Jonas disse. --- O que há de \emph{errado} com você?

--- Pelo menos o papai está bem --- se consolou a tia Beth.

--- Vivo. Bem \emph{mesmo} ele nunca está --- corrigiu a prima de Jonas.

Voltaram para casa pouco tempo depois. Já na manhã seguinte o avô de Jonas teve alta e pôde passar o Natal com a família. Eles não paravam de falar do menino Jesus. Era bebê Jesus isso, bebê Jesus aquilo.

``Jesus morreu como um cara barbado!'', Jonas pensava, ``Mas para eles é como se tivesse várias versões de Jesus. Como a Barbie. Tinha o Bebê Jesus, Jesus na Cruz. E os modelos especiais, Jesus e Maria Madalena, Jesus Ninja, Jesus Cowboy, Jesus \textsc{gls} e até mesmo Jesus Rockstar''.

Sempre rolava uma discussão entre algum dos casais. E sempre parecia ser por um assunto tão insignificante que Jonas se perguntava como eles podiam brigar por aquilo. E não eram pessoas ignorantes ou sem estudo. Seus tios eram pessoas instruídas. Mas mesmo assim brigavam por coisas estúpidas como deixar a torneira pingando ou ter derramado suco no chão.

Talvez aquela discussão aparentemente sem sentido não fosse o motivo da briga. Talvez fossem as frustrações acumuladas, a falta de compreensão ou carência, que, por uma coisa banal, funcionava como um agente catalisador, trazendo à tona o pior de cada um, mostrando todas as suas garras. Por que pessoas de excelente qualidade humana não conseguem dialogar? Onde foi parar o melhor de cada um deles?

Será que um dia ele também seria assim? \mudanca{Pensou} em Falls. Quando? O tempo todo. Ele sentia um desejo tão grande de vê-la. Mas como saber a diferença entre limerância e amor? E se ele começar a namorá-la, talvez se casar com ela, e um dia descobrir que tudo aquilo que \mudanca{sentia} simplesmente se foi?

Se você perguntasse \mudanca{lhe perguntasse o que faltava em sua vida há algumas semanas}, talvez \mudanca{Jonas} pudesse \mudanca{dar} uma lista enorme, porém\mudanca{,} no topo dela\mudanca{,} estaria uma namorada. Eles não estavam exatamente namorando, mas estavam bem perto disso. Agora que finalmente estava conseguindo tudo aquilo que ele sempre quis, suas pernas tremiam e ele não sabia o que fazer.

Quando ela não estava por perto era como se ele não se sentisse completo. Olhava os lugares vazios que \mudanca{Falls} poderia estar preenchendo. Ficava procurando-a nos supermercados ou andando pela rua. Ele até pensou em ligar para ela, centenas de vezes, mas não conseguia. \mudanca{Previa} que um dia ia terminar. Se até uma estrela morre, nossos relacionamentos também acabando quebrando. Pensando nisso\mudanca{,} ele se perguntava se valia a pena se jogar em algo fadado a acabar. Aquilo era amor, \emph{amor de verdade}? Eles iriam dançar a balada de~\emph{O}? Ou seriam felizes, plenamente felizes? Por mais que as perguntas preenchessem sua cabeça, sem nenhuma resposta aparente, ele sentia dentro de si aquela ânsia monumental, pantagruélica e gigantesca que o fazia pensar nos olhos de Falls e em cada pequeno gesto dela. Cada coisa que ela fazia era mágico. Em sua memória\mudanca{.} ganhava um contorno e um peso tremendos. Ali, rodeado de pessoas, ele se sentia sozinho. Mas bastava uma pessoa para se sentir completo.

Falls.

O que é amor, de qualquer forma? Alguém já escreveu que o amor é um velho fantasma de que todo mundo ouve falar, mas ninguém jamais viu. Jonas não pensa ser o caso. Na verdade, para ele o amor é como um sujeito cheio de falhas e defeitos, talvez até mesmo meio manco, mas que tem uma fama de herói grego, de atleta olímpico. O amor é subestimado. Estranhamente usamos a mesma palavra tanto para descrever o que sentimos pela nossa mãe quanto por uma mulher que damos um tapa na face. Embora Jonas nunca houvesse dado um tapa em nenhuma. Ele sempre fora solitário, o que implica \mudanca{em} dizer, buscava o sentido da vida. Agora\mudanca{,} quando olhava todos aqueles anos perdidos em lamentações e reflexões\mudanca{,} não passavam de uma sombra pálida perto daqueles momentos tão preciosos que tivera ao lado de Falls. Por mais nobres que fossem seus sentimentos, ele ainda era um prisioneiro de suas urgências biológicas. Sexo. O objetivo de todas as criaturas. A perpetuação da espécie. Para Schopenhauer a definição de amor é simples: ``um impulso sexual perfeitamente determinado e individualizado''. Estamos fadados a amar.

Então é isso o que somos, marionetes? De que adiantava saber tudo o que nos controla? Qual a diferença dele para qualquer outra pessoa? Era apenas uma marionete que podia ver as cordas. Nada mais. Por que ainda assim sentia-se impelido a uma direção, achando que daria certo? Isso o entristecia muito\mudanca{,} pois ele queria que desse certo, que fosse especial. Mas ele não teria garantias de nada. Era nisso que acreditava. Se você acredita, então não é mentira.

O amor é só um jogo. Acontece todo o tempo. Qualquer idiota no mundo ama alguém. Por que justo com ele seria diferente? Ele viajou de volta para casa tentando não se deixar tomar por esses sentimentos. \mudanca{Uma} vez \mudanca{havia lido} em uma revista que os terapeutas ensinam uma técnica para pessoas que estão passando por pesadelos recorrentes: que elas olhem para a palma de suas mãos. É um sinal libertador. Se elas fizerem isso em seus pesadelos, elas passam a ter o que se chama de sonho lúcido. Passam a controlar o que acontece e podem mudar todo o conteúdo do pesadelo, tornando-os doces sonhos. Neste momento, Jonas olhava para a palma da sua mão, mas sem sucesso. \mudanca{Não} conseguia controle algum.

Ele a amava. \mudanca{Decidiu} isso ainda dentro do ônibus. Se arrependimento matasse, coitado dele. Nunca havia dito ``eu te amo'' a ninguém. Lembrava quando foi a primeira vez que se deu conta disso. Já se fazem alguns anos. Foi no dia em que seu pai morreu. Eles sempre estiveram separados por um mar de silêncio. Ele deveria ter dito. Três simples palavras. Mas não. Transatlanticismo. Estava sempre tão perto, mas tão distante.

Quando Jonas chegou em casa, sentia-se melancólico. Ao abrir a porta sentiu a falta de Pixel\mudanca{,} que sempre se colocava na frente dela quando ele chegava. Foi direto ao telefone.

``Você tem sete novas mensagens'', ele ouviu da máquina.

Bip.

``Ei, Jonas, é a Falls. Eu fico ligando para a sua casa para falar com você. Mas eu sei que você não pode atender. Mas, mesmo assim eu continuo ligando. E isso é horrível. De qualquer jeito, estou entediada. Volte logo''.

Bip.

``Adivinhe só\ldots\,\emph{Sudoku}. Nível Médio. Tempo: dezoito minutos. Tome essa, espertinho''.

Bip.

``Novo Caminho, bom dia. Um instante, por favor. Desculpe, é que um dos estagiários estava parado aqui e preciso estar ocupada ou eles começam a me chavecar. Então, obrigada''.

Bip.

``Qual aquela palavra que a gente inventou quando temos alguma coisa presa no sapato? De qualquer jeito, eu tenho uma coisa presa no meu sapato''.

Bip.

``Pude sair mais cedo do trabalho, hoje é véspera de Natal. Estou ligando do meu celular. Estou indo para casa, porque eu tenho um gato para cuidar. Ligue quando voltar''.

Bip.

``Seu gato parece não gostar de mim, Jonas. Na verdade\mudanca{,} eu acho que ele me odeia. Ele sempre tem esse olhar de prepotência e desprezo? Porque é exatamente assim que ele está me olhando. Por que ele não fala? Poderia saber todos os seus segredos. De qualquer jeito, precisamos conversar. Muito''.

Bip.

``Quer saber como eu passei o ano? Comi todo um pote de sorvete e tomei quase uma garrafa de vinho. Deve dar para perceber pela minha voz, né? Ah! Seu gato escolheu a minha cama como vaso sanitário. Procuro pensar que isso significa que ele gosta de mim, Jonas. Eu não sei. Quero que você volte. Sei que não se diz uma coisa dessas por telefone, mas eu meio que estou apaixonada por você. Não quero ficar com joguinhos do tipo `eu finjo que sou perfeita, você finge que é perfeito'. Eu acho que eu estou apaixonada. Você deve estar achando que eu sou uma maluca ou algo do tipo. É isso. Me liga. Até''.
