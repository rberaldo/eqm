\chapter{Onde está Satã?}

Na manhã de segunda\mudanca{,} George estava radiante, com um sorriso que o fazia parecer o gato de Cheshire. O que irritava Jonas profundamente. Ele odiava ver as pessoas intoxicadas de alegria enquanto ele estava se achando o mais miserável de todos.

--- Como foi a festa? --- ele perguntou.

--- A melhor de todas --- respondeu George.

--- O que aconteceu?

--- Irrelevante.

Jonas arqueou as sobrancelhas e voltou a trabalhar, ou melhor, pensou em trabalhar, pois tão logo se dispôs a começar uma de suas tarefas, George o interrompeu com uma pergunta:

--- Como foi a sua noite?

--- Eu vi uma garota linda. Não consigo tirar ela da minha cabeça. Acho que estou apaixonado --- falou Jonas desolado.

--- Você é tão fraco, Jota --- reprimiu-o George. --- Além do mais, isso não é amor.

--- Então o que é? --- quis saber Jonas.

--- Sabe quando você vê uma garota na rua, e tudo faz sentido? Você ``fica caidinho por ela'' --- falou ele desenhando as aspas no ar. --- Você não sabe o nome, de onde veio ou para onde vai? Ou você subitamente se interessa por alguém que conhece. Não é amor. Como chamar isso? Uma americana estudou esse sentimento e chamou-o de \mudanca{\emph{limerence},} ou limerância. É um estado cognitivo emocional involuntário. É a isso que estamos nos referindo quando alguém tem uma ``queda por alguém''. Você têm pensamentos intrusivos. Vívida imaginação com o objeto desejado. Medo da rejeição, vergonha na presença do objeto. Deixa outras preocupações em segundo plano. Sensibilidade aguda para qualquer coisa que o objeto fale, pense ou faça, \mudanca{as quais possam ser favoravelmente} interpretadas\mudanca{,} e a extraordinária habilidade para inventar explicações ``razoáveis''  para descobrir mesmo em ações neutras um sinal de paixão no objeto. Ela se sustenta e desenvolve em certo balanço entre certeza e incerteza.

--- Uau --- Jonas disse. --- É quase uma doença mental.

--- Pode ter certeza --- falou George inflado. --- É uma obsessão cognitiva. Sexo com o objeto não é essencial ou suficiente para um indivíduo com limerância.

--- Quando você diz sexo com objetos soa como se alguém estivesse comprando algum brinquedo de sex shop --- Jonas observou levando os dois ao riso.

Jonas contemplou um tempo o vazio enquanto mexia uma de suas miniaturas sem qualquer interesse nela. Então se virou para George e disse:

--- Quando você falou isso\ldots\,Não sei. Eu senti como se nunca tivesse amado. Entende? Eu realmente me identifiquei com a limerância. Parecia ser a descrição de tudo o que antes eu já chamei de amor.

--- Então, o que você acha da Regina?

--- Não sei, cara --- Jonas disse confuso. ---  O que isso tem a ver com o assunto?

--- Nada. Só estou fazendo uma pergunta. O que você acha?

--- Você está interessado nela? --- Jonas perguntou como se isso fosse algum contra senso.

--- Irrelevante, Jota. Limite-se a responder minha pergunta.

--- O que você acha dela? --- Jonas disse\mudanca{,} depois de dar de ombros.

--- Ela é religiosa --- falou George\mudanca{,} mudando o tom de sua fala.

--- E isso é um problema? --- perguntou Jonas.

--- Claro que é --- desdenhou George. --- Você sabe por quê\ldots

George sorriu.

--- Elas\ldots\,--- Jonas não tinha ideia nenhuma. Em se tratando de George\mudanca{,} poderia ser qualquer coisa. Literalmente qualquer coisa --- São religiosas --- afirmou incerto.

--- Bem, isso é obviamente um problema --- afirmou George como se surpreendido pela resposta ---  Não deixa de ser. Mas outro --- como se houvesse alguma coisa óbvia que estivesse escapando a Jonas.

--- Elas não fazem sexo --- chutou.

--- Mito --- George disse com um estranho sorriso no rosto.

--- Eu desisto. Eu realmente não sei George. --- falou Jonas levantando os dois braços, rendido.

--- Se elas engravidam, elas continuam assim --- ele disse\mudanca{,} com um daqueles sorrisos de superioridade\mudanca{,} que praticamente é o único que ele sabe fazer.

--- Por que você me perguntou justo sobre a Regina? --- Jonas lhe perguntou curioso.

--- Irrelevante --- ele disse voltando a digitar. --- Quer ir até a livraria comigo na hora do almoço? Chegou um livro novo que eu quero comprar.

--- Eu não posso, tenho um compromisso --- falou Jonas.

--- E seu disser que eu vou comprar o novo livro de Neil Gaiman --- George disse com satisfação. --- Isso faria você ignorar esse tal ``\emph{compromiso}''? --- ele concluiu\mudanca{,} desenhando \mudanca{novamente} aspas no ar com os dedos e dizendo a palavra de forma irônica.

--- É sério. Eu realmente tenho um compromisso, George.

George riu e disse:

--- Você nunca tem um compromisso.

--- Hoje deve ser dia de São Nunca então, por que eu tenho um compromisso. Eu marquei uma consulta na Novo Caminho.

--- O quê? Você vai ir mesmo nesse lugar? --- George protestou --- Você pode morrer! É como brincar de roleta russa. Você entende? Você pode morrer de verdade. Você realmente entende o que morte significa? Não dá para dar \emph{reset}. Ou continuar de onde você tinha salvado. \emph{Game over}.

--- Eu sei George. Mas eu já fiz minha escolha. Não é problema seu --- falou Jonas.

--- Tudo bem. Não vou dizer mais nada --- disse George enquanto digitava freneticamente --- Eu só me importo com você, só isso.

Jonas ouviu aquilo e se sentiu culpado. Mas não muito. E voltou a fazer o seu trabalho até poder sair para o consulta que iria fazer. Enquanto passava pela porta, George lhe disse:

--- Boa Sorte, Jota.

--- Obrigado, George.

--- Vá atrás de seus sonhos --- ele disse com a boca cheia de salgadinhos. Jonas se virou para ele e sorriu. --- E quando voltar me traga uma soda, por favor.

--- Tudo bem, George. Até.

Ele não demorou muito para chegar até a Novo Caminho. Ficou com medo de não conseguir encontrar o lugar ou chegar atrasado, mas foi relativamente simples. Era um prédio branco com um pequeno jardim cheio de flores coloridas. Um tapete verde que lembrava visualmente os melhores gramados de estádios de futebol. No meio do jardim havia um pequeno caminho de tijolos amarelos por onde Jonas caminhava e olhava para as letras e logo de metal que foram colocados em uma parte do jardim. Lá estava o Ankh que antes atraíra sua atenção para apanhar o folheto.

Ao chegar perto das portas, elas se abriram automaticamente. Fazia um pouco de frio lá dentro em comparação com o dia ensolarado de lá fora. Havia um balcão bem próximo da porta e sofás em uma sala de espera. Havia apenas duas pessoas sentadas lá. Um senhor de idade usando uma bengala e uma garota bem jovem, de cabelos negros e olhos que mais pareciam duas bolas de piche. Usava uma roupa igualmente preta e se concentrava em um computador onde digitava alguma coisa.

Morte é definida como a cessação completa e irreversível dos processos físicos que sustentam o fenômeno da vida biológica. Mas fica um pouco mais complicado quando se descobre que eles não possuem uma definição precisa sobre o que é vida. Biologicamente, a morte pode ocorrer para o todo, para parte do todo ou para ambos. Por exemplo, é possível para células individuais, ou mesmo órgãos morrerem, e ainda assim o organismo como um todo continuar a viver. Muitas células individuais vivem por apenas pouco tempo, e a maior parte das células de um organismo são continuamente substituídas por novas células.

Também é possível que o organismo morra (geralmente, num caso de morte cerebral) e que suas células e órgãos vivam, e sejam usadas para transplantes. Porém, neste caso, os tecidos sobreviventes precisam ser removidos e transplantados rapidamente ou morrerão também. Em raros casos, algumas células podem sobreviver, como no caso de Henrietta Lacks. Suas células literalmente são imortais e se tornaram até mesmo uma nova espécie por terem sobrevivido décadas após sua morte. Foi graças às células imortais de Henrietta que a vacina contra a Poliomelite pôde ser criada.

A irreversibilidade é constantemente citada como um atributo da morte. Cientificamente, é impossível trazer de novo à vida um organismo morto. Se um organismo vive, é porque ainda não morreu anteriormente. No entanto, muitas pessoas não acreditam que a morte física é sempre e necessariamente irreversível, enquanto outras acreditam em ressuscitação do espírito ou do corpo e outras ainda têm esperança que futuros avanços científicos e tecnológicos possam trazê-las de volta à vida, utilizando técnicas ainda embrionárias, tais como a criogenia ou outros meios de ressuscitação ainda por descobrir.

Mas o que acontece a uma pessoa, ao indivíduo, quando ele morre? Existe uma alma que após passar pelo clichê do túnel de luz vai para o Céu ou Inferno dependendo de suas escolhas morais em vida? Iremos reencarnar em uma próxima vida, esquecendo tudo aquilo que vivemos anteriormente? Quando você morre e entra na roda da dança macabra, a rigidez cadavérica inicia-se umas quatro horas depois do cessar completo das funções vitais, começando na mandíbula inferior e na nuca, e concluindo nas pernas, prolonga-se até dois ou três dias após o instante do óbito. Se você é um homem e morreu em uma posição ereta ou de bruços é provavelmente que você esteja com um ereção. Uma ereção depois de morto. Com a morte, cessa-se a circulação do sangue, o que o faz se concentrar nas regiões mais baixas do organismo por causa da gravidade.

Com a rigidez do defunto, podemos dar por encerrada qualquer esperança de reanimação. O terrível destino que lhe espera agora, a aniquilação por decomposição química ou putrefação de sua estrutura celular, reduzirá a zero a debilísssima expectativa de reanimação. Em nenhum lugar de nosso mundo físico pode-se comprovar melhor que em um cadáver como a batalha final entre dinamismo e entropia acaba sempre sendo ganho pela última. O campo desse cenário épico onde podemos assistir à desigual luta entre Tanatos e Eros é o cadáver do homem. Após a morte, a autodestruição dos tecidos celulares inicia a sua macabra atividade. E mais tarde, os fungos, micro-organismos necrófagos que não perdoam nada nem ninguém, iniciarão também seu voraz banquete.

Em poucas horas\mudanca{,} os órgãos mais delicados de nosso corpo ficam reduzidos a uma massa viscosa e pestilenta. A posição medular das glândulas suprarrenais amolecem, convertendo-se em uma cavidade imunda que segrega um líquido ligeiramente pardo. As paredes do estômago e os intestinos, sem a renovação sempre constante de suas células, por autodigestão, tornam-se totalmente moles. Os sucos gástricos, que até agora haviam respeitado os receptáculos que os continham, perfuram-nos e vão derramando-se pelas cavidades peritoneais.

A cavidade pleural, junto ao pulmão, que contém uma substância sumamente ácida, ao reagir frente aos líquidos gástricos que avançam através do diafragma, inicia uma ação duplamente destruidora sobre o aparelho respiratório. O último que resta por intervir são as bactérias putrefatas que o farão, quando os primeiros agentes químicos lhe abrem uma brecha. A destruição das vísceras chega a níveis tão arrasadores que custaria descrevê-los. Não há estômago (vivo) algum que aguente muito tempo devido ao cheiro nauseante e fétido (do estômago morto). As parênquimas são aniquiladas até liquefazerem-se. O fígado transforma-se numa repulsiva substância esverdeada-escura, e o cérebro, essa maravilhosa estrutura de onde a humanidade tirou seus pensamentos e toda suas grandes maravilhas e monstruosidades, onde obras de arte foram concebidas, reduz-se a uma massa amorfa verde-cinza e viscosa. É o fim. Isso é tudo o que a ciência pode dizer que acontece após a morte.

Se Jonas debate-se tanto, é porque sinceramente não sabe ou não gostaria de acreditar que isso é tudo. Acreditar em uma existência após a morte não é um pensamento otimista ainda que ingênuo, e talvez errado? Talvez aquele cara de \emph{Matrix} esteja certo, a ignorância é uma bênção. Mas Jonas queria ter mais que uma intuição ou crença. Ele queria ver. Com seus próprios olhos.

Jonas voltou sua atenção para o balcão e viu a recepcionista que estava com um livro enorme aberto à frente dela. Ela olhou para ele e suas sobrancelhas se arquearam.

--- Boa tarde --- ela disse.

--- Boa tarde --- ele respondeu. --- Eu tenho uma hora marcada com o Dr.~Roberto.

--- Nome?

--- Jonas.

--- Deixe-me checar --- ela se virou para o computador. Jonas tentou ler o nome dela que estava em um crachá.

--- Tamanho~42 --- ela disse.

--- O quê? --- perguntou Jonas sem entender.

--- Meus seios --- ela disse colocando as mãos sobre eles.

--- Oh, não\ldots\,É\ldots\,Não foi isso --- ele não sabia como começar a se explicar e\mudanca{,} o pior, agora não conseguia evitar de olhar para eles.

--- Pode olhar\mudanca{,} eu tenho orgulho deles.

--- Não, eu só queria ler seu nome, no crachá.

--- Que horas são? --- ela perguntou. Então ele se deu conta por que ele achou a garota na festa com uma voz familiar. Por que ela estava ali bem na frente, era a recepcionista da Novo Caminho.

--- Nós nos vimos na festa? --- ele perguntou.

--- Nós nos vimos na festa? --- ela repetiu a pergunta dele em um tom neutro.

Jonas estava claramente nervoso. Seu impulso era dar meia-volta e voltar correndo para o carro e ir para bem longe esconder a cabeça debaixo de um buraco de terra.

--- Estou apenas brincando --- ela disse abrindo um sorriso. ``Covinhas'', Jonas pensou.

--- Sr.~Michael Scott --- ela disse chamando a atenção do homem com a bengala. --- O Dr.~Patrick o aguarda na sala dele. É a primeira à direita.

O homem fez um sinal de agradecimento com a mão e se levantou. Jonas olhava para ela. Nem mesmo conseguia olhar nos olhos dela. Foi quando uma voz atrás dele, vinda de fora, gritou:

--- Onde está Satã? Onde está Satã?

A expressão dela de simpatia se transformou em preocupação. Ela abriu uma gaveta e tirou algo de lá. Jonas viu que seu nome era Falls. O homem era enorme e sua face estava vermelha de raiva. Ele é realmente o tipo de cara que ninguém gostaria de estar em seu caminho.

--- Sr.~Tomás, por favor, afaste-se ---  Falls disse saindo balcão.

--- Eu vim conversar com aquele sujeito --- ele disse para ela --- Você não têm nada a ver com isso. Ele começou a dar passadas largas em direção ao corredor.

--- Pare! --- Falls disse ao mesmo tempo em que espirrou alguma coisa de um tubo preto que ela tinha tirado da gaveta. Rapidamente deu para perceber que era spray de pimenta. Mesmo tendo acertado diretamente no rosto do sujeito com raiva, Jonas sentiu arder seus olhos e sua boca. Começou a esfregar os olhos. Falls começou a tossir. Logo, a garota de preto havia chamado alguém de jaleco branco, que Jonas somente podia ver através de seus olhos embaçados pelas lágrimas que começavam a sair.

--- Meus olhos! --- Jonas falou. Ele tinha que estar bem atrás do sujeito que leva um jato de spray de pimenta.

--- O que aconteceu? --- perguntou uma voz vindo do corredor. Jonas mal conseguia abrir os olhos que ardiam. E ouvia um burburinho de pessoas ao seu redor.

--- Tomás\mudanca{,} de novo --- disse uma voz.

--- Já chamei a polícia.

--- Levem a Falls e aquele cliente para passarem água no rosto.

Após isso Jonas sentiu alguém o guiando. Quando ele passou água no rosto e conseguiu abrir o olho viu que estava em uma sala parecido com um consultório. Falls estava com o rosto molhado e parecia lacrimejar.

--- Eu achava que só afetava o cara mau --- disse Falls, sentindo-se culpada.

--- É um spray. A tendência é que ele se espalhe para todos os lados, como um desodorante ---  disse um sujeito jovem antes de sair rindo em alto e bom som. O rapaz saiu da sala, deixando Jonas e Falls a sós. Durante um tempo houve aquele silêncio desconfortável entre duas pessoas que pouco se conhecem.

--- Desculpe --- Falls disse ao mesmo tempo em que começou a rir. Jonas riu junto com ela.

--- Tudo bem --- ele disse passando mais uma vez as mãos nos olhos.

Ela colocou a mão em seu ombro.

--- Não foi por mal --- ela disse. Ele se sentiu bem com a mão dela em seu ombro, respirou fundo e quis aproveitar cada segundo daquilo que não durou muito, só até a porta se abrir e entrar um homem de jaleco, distinto e com um ar contemplativo. Seus olhos de coruja pareciam olhar através das pessoas, como se para ele não houvesse segredo nenhum ao qual eles não tivessem acesso.

--- Desculpe pelo inconveniente --- ele falou se dirigindo a Jonas. --- Ele está em crise desde que a mulher morreu e nos responsabiliza por isso. Eu o entendo de certa forma --- ele concluiu em tom mais baixo e não olhando para lugar algum em especial.

Ele se virou para Falls e deu uma boa olhada nos olhos dela.

--- Você vai sobreviver. Bom trabalho! --- ele falou sorrindo. --- Pode me acompanhar até a minha sala? --- disse se virando para Jonas.

--- Claro --- este respondeu.

Ele saiu da sala e Jonas começou a acompanhá-lo.

--- Este é o doutor\ldots\,--- sussurrou Jonas para Falls.

Ela fez um sinal de afirmação com a cabeça e sorrindo disse:

--- Satã, em carne e osso.

Ele saiu da sala, e viu Satã, ou melhor, Dr.~Roberto parado ao lado da porta em que deveria entrar.

--- Vamos começar de novo --- disse o Dr.~Roberto. --- Sou o Dr.~Roberto Mouir.

Jonas disse seu nome e eles trocaram um aperto de mão. O Dr.~Roberto pediu que ele entrasse, quando o fez, fechou a porta.

--- Queria pedir novamente desculpa pelo ocorrido --- falou Dr.~Roberto.

--- Tudo bem --- Jonas disse. --- Aquele homem estava realmente nervoso.

--- Não é todo dia que um cliente insatisfeito vêm reclamar --- disse Dr.~Roberto se sentando. --- Na verdade quando nós cometemos algum erro eles não podem mais reclamar, é claro\ldots

Jonas ficou mudo.

--- Mas nós raramente cometemos erros --- Dr.~Roberto disse tentando consertar o que havia acabado de dizer. --- Então você está interessado em fazer uma \textsc{eqm}?

--- Exato --- respondeu Jonas. --- Bem, na verdade estou um pouco hesitante ainda.

--- Natural. Faz parte. Tem alguma pergunta?

--- Eu queria fazer uma pergunta, qual é a chance de que eu\ldots\,Ah\ldots\,--- falou Jonas sem conseguir pronunciar a palavra.

--- Morra? --- perguntou Dr.~Roberto curto e sem surpresa. Talvez por já estar acostumado com esse tipo de dúvida.

--- Isso --- disse Jonas engolindo em seco.

--- Tecnicamente, você vai morrer com certeza. Nós te trazemos de volta. Mas, a respeito de não conseguirmos fazer isso, ou seja, te ressuscitar\ldots\,Digamos que fazer esse procedimento é mais seguro que viajar de automóvel por exemplo.

Tranquilizador. Dizem a mesma coisa sobre aviões, mas isso não significa que acidentes não acontecem.

--- Como é feito?

--- É um procedimento patenteado chamado \emph{Método Ars Moriendi}. Basicamente nós fazemos uma asfixia química, que faz com que você tenha uma parada cardiorrespiratória. Trazemos você de volta com um desfibrilador. Se você quiser realizar o procedimento terá que fazer toda uma bateria de exames. Há vários pacotes, com diversos preços --- falava Dr.~Roberto enquanto pegava um pequeno libreto. --- Leve para casa, estude com cuidado. E caso queira, e somente caso queira, você poderá marcar um procedimento pelo telefone.

--- Tudo bem --- disse Jonas pegando o folder.

--- Sente-se na mesa de exames, por favor. Mesmo que você não chegue a fazer, pelo menos vai ter passado por uma pequena avaliação física. Alguma alergia?

Após dez minutos Jonas já estava saindo do consultório e viu dois sujeitos andando lá dentro passando pelo corredor.

--- Você encomendou mais Pavulon, certo? --- um deles disse.

--- Claro. Senão Patrick ia me comer vivo, de novo --- respondeu o segundo.

--- Seus olhos melhoraram, cara? --- perguntou um deles dando uma olhada no rosto de Jonas o que o deixou sem graça.

--- Melhorou sim, obrigado.

Eles continuaram seu caminho enquanto Jonas se voltava para o outro lado, mas ele ainda foi capaz de ouvir um deles comentando:

--- O que uma garota com um spray de pimenta pode fazer!

Quando ele chegou novamente até a recepção, lá estava a garota toda vestida de preto agindo novamente como se não existisse mundo exterior e Falls, ainda com parte do rosto avermelhada, assoava o nariz em um lenço.

--- Tchau --- ele disse bem baixinho\mudanca{,} fazendo um sinal com a mão.

--- Tchau! --- ela disse\mudanca{,} abrindo um sorriso.

\begin{sloppypar}
Durante o caminho de volta para o trabalho, Jonas não conseguia a tirar de sua cabeça. ``Eu estou completamente apai\ldots\,Não. Eu estou completante limerente'', ele pensou.
\end{sloppypar}

Resolveu ligar o rádio, estava tocando ``\mudanca{\emph{Everybody Hurts}}'' do REM. Ele desligou na hora, não queria ouvir músicas tristes para não ficar assim o resto do dia.

``Eu sou uma guerra de cabeça contra coração'', Jonas concluía a respeito de si mesmo. ``E é sempre do mesmo jeito. Minha cabeça é fraca, meu coração sempre fala antes mesmo de eu saber o que eu vou dizer''.

Será que existia algo como escolha racional? Todas as ações dele\mudanca{,} em retrospecto\mudanca{,} parecem ser todas tomadas ou por impulso ou postergadas até o momento final\mudanca{,} quando são decididas novamente por impulso. E isso se baseando apenas em suas emoções. Talvez aconteça o mesmo com outras pessoas. Homens e mulheres só se guiam pelo coração. A diferença básica é que muitas vezes o coração em um homem se localiza entre suas pernas.

--- O que aconteceu? --- perguntou George surpreso para Jonas quando este entrou na sala onde trabalhavam. --- Olhe para você --- disse George com entusiasmo. --- Parece que chorou tanto quanto uma garotinha.
