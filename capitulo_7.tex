\chapter{Vejam vocês mesmos}

O Modelo Kübler-Ross tenta descrever os estágios pelos quais uma pessoa passa após sofrer grandes perdas e tragédias.

Negação é o primeiro. Ele vem logo depois que você recebe a notícia. É como uma proteção, uma fuga da realidade. Na prática funciona tanto quanto desviar o olhar de algo. Não importa por quanto tempo você permaneça assim, isso não vai mudar o que aconteceu. Mesmo sendo médico a vários anos, Roberto Mouir não tinha ideia de como seria estar do outro lado. Inúmeras pessoas morreram em suas mãos, ele fez de tudo para salvá-las. Sofreu um pouco por cada uma, mas nada foi comparado quando ele se viu do outro lado.

Ele chegou de um seminário em que havia se ausentado há três dias. Havia conversado com a esposa quando saía de lá. Elas já deveriam estar há um bom tempo esperando por ele. Ligou várias vezes e não teve sinal delas. Ficou com muita raiva. Ele estava irritado há alguns dias. Era somente uma fase ruim.

Quando ele viu seu cunhado, Patrick\mudanca{,} chegando apressado com o olhar esquadrinhando as pessoas, ele começou a tremer. Ele foi em sua direção, pressentia a má notícia.

Patrick ficou em sua frente sem conseguir falar. Ele começou a chorar. Roberto passou a mão na cabeça e não conseguiu controlar o desespero.

--- Larissa morreu --- falou Patrick\mudanca{, abraçando-o}.

--- Eu não estou aqui. Isso não está acontecendo. Eu não estou aqui. Isso não está acontecendo --- recitava baixo Roberto como que lançando um feitiço. Mas nada aconteceu.

Eles se conheciam desde a faculdade de medicina. Dividiram um quarto junto\mudanca{s}. Eram a família um do outro. Quando conheceu a irmã de Patrick, Roberto oficialmente se tornou parte da família Kafka.

--- Como foi? --- perguntou Roberto tentando se controlar ao se sentar em um banco do aeroporto. Ele olhava para os lados e via olhos curiosos mirando sua pessoa. A opinião das pessoas sempre exerceu nele uma profunda preocupação, mas agora ele não conseguia sequer pensar nisso.

--- O carro delas se chocou contra um motorista bêbado --- falou Patrick.

--- Sarah! --- exclamou Roberto\mudanca{,} como se somente agora se lembrasse que tinha uma filha de nove anos.

--- O médico disse que foi um milagre. \mudanca{Ela} vai ficar bem --- ele disse colocando a mão no ombro de Roberto.

``Eu não consigo acreditar'', dizia Roberto a si mesmo, enquanto seus olhos em vão vasculhavam o lugar atrás de sua mulher, como se ela pudesse chegar a qualquer momento. E aquilo simplesmente seria esquecido, talvez rissem daquilo algum dia. Mas ela não veio. Nunca mais voltou. Mas seus olhos nunca pararam de procurar.

O segundo estágio no modelo~Kübler-Ross é a raiva. Ele sabia que não havia esperança nenhuma, mas correu ao hospital após se recuperar do choque. E quis ver o corpo da esposa. Tudo o que \mudanca{se} lembrava era de que ela tinha ligado para ele antes de sair de casa. Ela tinha essa mania de ligar sempre para ele. Ele gostava. Era quando ele realmente se sentia amado. Mas ele não estava passando por uma fase muito boa, na verdade ele estava irritadiço e descontava em todos ao seu redor. A convenção havia sido estressante. Ele teve alguns problemas para conseguir sair do hotel e \mudanca{pegar} o avião, saiu realmente atrasado. Quando o celular tocou ele já atendeu bufando de raiva. Mesmo com um milhão de desculpas para estar do jeito em que estava, nada disso o inocentaria diante de seus próprios olhos.

--- O que foi? --- ele perguntou impaciente.

--- Só queria saber se está tudo bem --- falou Larissa.

--- Estaria se você não ficasse me ligando. Estou atrasado --- ele disse, desligando o telefone.

Agora ela estava morta. Sentia raiva de si mesmo por ter sido tão estúpido. Não havia volta. ``Por que isso está acontecendo?''\mudanca{, pensava} Roberto. ``A vida não é justa. Eu posso viver com o fato de que a vida é injusta. Mas que fosse injusta ao meu favor!''.

Ele sentia em si pulsando uma raiva que queria sair de dentro de si. Ele chegou a chutar a parede, quase torcendo o pé. Quando saiu da sala, não sabia o que fazer. Simplesmente desmoronou novamente e começou a chorar. ``Por quê?'' se perguntava, ``Por quê?''. Chorando, sentiu um braço tocá-lo no ombro. Se virou e viu o rosto de um homem de cavanhaque, usando óculos e que devia ter a sua idade. Um médico. Roberto passou a mão no rosto, limpando as lágrimas. Não sabia quem o homem era. Mas sentiu raiva dele. Apenas por ser um médico e estar do outro lado. Ele não sabia o que ele estava passando. Não fazia a mínima ideia pelo que ele estava passando.

--- Sr.~Roberto? --- perguntou o médico.

--- Sim? --- Roberto disse com a voz um pouco rouca.

--- Eu sou o médico que atendeu a sua filha quando chegou aqui. Eu acho que o senhor gostaria de saber --- ele dizia com uma espécie de satisfação que alimentou ainda mais o ódio que Roberto sentia --- que o que aconteceu com sua filha\ldots\,Foi um milagre. Foi uma obra de Deus.

--- Deus? --- perguntou Roberto colocando toda sua ira na palavra. --- Deus não existe. Onde está Deus? Hein? --- ele disse saindo de perto do homem e apontando para a sala onde estava o corpo de sua mulher --- Onde está Deus agora?

O médico apenas \mudanca{ficou-o} observando\mudanca{,} sem conseguir esboçar nenhuma palavra. Em seu rosto estava estampad\mudanca{a} uma espécie de frustração. Como se a notícia que ele trouxesse de certa forma fosse confortar Roberto.

--- O mundo --- disse Roberto --- deve ser entendido como resultado do caos e sorte cega. E se provir de um propósito deliberado\ldots\,Esse só poderia vir de um demônio.

Roberto deu as costas ao homem e caminhou em direção ao próximo estágio de Kübler-Ross, a barganha.

Ele sentia-se imensamente frustrado com a morte da mulher e acabou negligenciando não apenas o emprego, faltando dias seguidos, mas até com sua própria filha. Somente Patrick conseguia mensurar o quanto Roberto foi destruído por aquilo. Roberto começou a beber, a perder peso e a abusar de antidepressivos que ele receitava a si mesmo. Patrick não poderia ver o amigo se destruir sem dizer nada.

--- Roberto, você têm uma filha para criar\ldots\,--- começou Patrick.

--- Eu sei que eu tenho uma filha para criar, você não precisa me lembrar disso --- ele falou irritado, interrompendo Patrick --- Por quê, Patrick\ldots\,Por quê? Eu preferia que Sarah tivesse morrido no lugar dela --- e começou a chorar. Patrick olhou para ele imaginando se ele realmente queria ter dito o que acabou de dizer.

--- Eu vou me mudar para sua casa --- Patrick disse. Roberto apenas balançou a cabeça concordando\mudanca{,} ao contrário do que Patrick esperava.

Logo na primeira noite, ao se levantar para tomar um copo de água, Roberto estava sentado em uma cadeira com os olhos brilhantes.

--- Não consegue dormir? --- perguntou Patrick\mudanca{,} pensando que talvez já era hora de Roberto pensar em consultar um psicólogo.

--- Você não tem ideia do que acabou de acontecer --- Roberto disse\mudanca{,} olhando para o vazio. Depois\mudanca{, olhando nos olhos} de Patrick como há muito tempo ele não fazia, como se daquela vez ele realmente estivesse ali presente, não somente a sua casca, como agira todo o tempo desde a morte de Larissa. Ele soltou uma espécie de riso e falou:

--- Nem mesmo eu consigo acreditar. Quero dizer, você não vai acreditar em mim. Vai dizer que eu estou louco.

--- Tente --- Patrick disse.

--- Eu estava me revirando na cama. Há muito tempo que eu não consigo dormir. Fico o tempo todo me virando de um lado para o outro. O inferno na terra. O quarto ficou extremamente frio. Como estamos no verão eu não estava com um cobertor na cama, apenas com um lençol vermelho. Fiquei incomodado com o frio ao ponto de começar a me levantar para pegar um cobertor no armário. Mas eu nem cheguei a fazer isso. Eu vi a Larissa no meu quarto. Bem, não era exatamente como a Larissa, mas era ela de alguma forma. Sua pele estava em um tom azul pastel claro. Os lábios vermelhos como sangue e olheiras profundas. Eu senti tanto fascinação como repulsa. Tudo o que eu queria era ver novamente minha mulher e quando isso acontece eu fico apavorado. Eu devo estar começando a delirar.

Patrick sentiu um calafrio percorrer todo o seu corpo, colocando cada pelo de seu corpo em pé. Ele não era particularmente religioso, mas tinha um lado espiritual. Larissa certamente o tinha também.

--- O que você acha? --- perguntou Roberto.

--- Você a viu. E o que aconteceu em seguida? --- Patrick mostrou-se interessado.

--- Ela me disse: ``\emph{Vejam vocês mesmos}''.

--- ``Vejam vocês mesmos?" Você têm ideia do que isso quer dizer?

--- Lembra-se que antes mesmo de nós namorarmos, uma noite que ela estava te visitando e nós três acampamos e no meio de uma conversa fizemos uma espécie de promessa? O primeiro que morresse deveria voltar e contar aos outros como é o outro lado. Eu acho que ela veio cumprir a promessa. Desde o momento em que a vi, eu sabia que devia ser isso.

Eles ficaram se olhando por um tempo. Roberto tinha perdido a gravidade que carregava no olhar. Parecia que ele tinha tirado um peso enorme dos ombros. Sua face não estava tão rígida como antes.

--- Quando Larissa morreu eu cheguei a lembrar dessa promessa. Mas eu não quis tocar no assunto --- falou Patrick\mudanca{,} sem jeito.

--- Eu não sei. O que você acha? --- lhe perguntou Roberto.

--- Você estava certo --- falou Patrick --- Você está louco. E precisa de uma boa noite de sono.

--- Nisso nós dois concordamos --- Roberto disse se espreguiçando na cadeira.

--- Sobre você estar louco ou precisar de uma boa noite de sono?

--- Os dois --- disse Roberto permitindo-se sorrir pela primeira vez em semanas.

Nos dias que se seguiram ele começou a se mostrar mais animado. Voltou a trabalhar, mas dedicava um bom tempo a fazer pesquisas, comprou um monte de livros com nomes estranhos como \mudanca{\emph{Bardo Thödol}, \emph{Necronomicon} e \emph{Mors ontologica}}. Estava obcecado com o além-vida e dedicava seu tempo a estudar a visão da maioria das culturas sobre o assunto.

Um dia\mudanca{,} quando Patrick jogava banco imobiliário com Sarah, Roberto não se aguentando de excitação, interrompeu-os durante o jogo para dizer algo a Patrick.

--- \mudanca{Já volto}, e sei exatamente quando dinheiro você tem --- Patrick disse apontando para as notas.

Ele se aproximou de Roberto que ainda nem mesmo havia retirado o jaleco branco.

--- Então? --- perguntou Patrick, cruzando os braços.

--- Eu sei o que Larissa quis nos dizer. Eu sei --- falava Roberto sentindo uma pulsante alegria dentro de si.

Patrick o observava, com certo receio.

--- Ela quer que nós vejamos o outro lado com nossos próprios olhos --- ele disse eufórico como se tivesse acabado de descobrir ouro.

--- E como você espera fazer isso, Roberto? \emph{Se matando}? --- Patrick perguntou-lhe sarcasticamente.

Eles se olharam por alguns instantes. Patrick ficou preocupado com o que Roberto poderia fazer. Olhou para Sarah e temeu que algo pudesse acontecer com ela. O que Roberto tinha em mente? Não bastava ter perdido a mãe tão repentinamente, agora o pai estava se afundando em paranoia.

--- Nós podemos voltar --- Roberto disse em um tom mais baixo. --- Já ouvir falar em \textsc{eqm}?

--- Experiências de quase-morte? --- perguntou\mudanca{, surpreso,} Patrick. --- O quê? Como você espera fazer isso? É loucura, Roberto. Loucura.

--- Eu ainda não sei, mas vou descobrir --- Roberto disse\mudanca{,} apontando para \mudanca{seu cunhado} antes de se virar e sumir em um corredor.

Patrick ficou olhando de onde estava para Sarah. Ela parecia estar lidando com a morte da mãe de uma forma melhor do que Roberto. Roberto estava tentando negociar com a morte. Tentando de alguma forma dominá-la e descobrir o seu mistério. Domesticá-la. Patrick desejou que ele simplesmente deixasse isso de lado, mas não foi o que aconteceu.

\begin{sloppypar}
Roberto mergulhou em uma profunda pesquisa e depressão. Passou a fazer plantões de quarenta, cinquenta horas. \mudanca{Quando em} casa, lia livros imensos e acabava dormindo com eles em seu colo.
\end{sloppypar}

--- Eu descobri uma forma --- ele disse para Patrick no meio de um jantar, mudando completamente o assunto da conversa. Patrick olhou para Sarah.

Não sabia o que fazer, mas\mudanca{,} mesmo achando terrivelmente errado\mudanca{,} acabou concordando com ele.

--- Eu não entendi o mesmo que você Roberto, que isso fique bem claro --- ele disse mais tarde enquanto lavavam louça. --- O que eu acho que ela quis dizer é que talvez os segredos dos mortos não são para os vivos. Porque ela disse, ``Vejam vocês mesmos''. Isso quer dizer, não sei, é o meu palpite, que cada um deve ver com os próprios olhos, na sua vez. Isso não parece estar certo. É loucura. Mas eu vou ajudá-lo, confesso que estou curioso. Quem não estaria? E que fique bem claro que você e eu naquela noite estávamos certos. \emph{Você está louco}.

Demorou muito para que Roberto encontrasse uma cobaia para o primeiro experimento. Um sujeito que tinha aulas de sonhos lúcidos com ele e que ansiava por experiências místicas, e não parecia nenhum pouco preocupado com os riscos. \mudanca{Sua face poderia se comparar à de} uma criança que estava indo para a Disney. Patrick chegou a comentar com Roberto o fato, achando isso muito estranho.

--- Há milhares como ele lá fora, Patrick --- falou Roberto em um tom frio.

--- Mas você contou a ele exatamente o que vamos fazer? --- Patrick lhe falou.

--- Não e nem devemos. Fazer isso seria matar a mágica, Patrick. É nosso pequeno segredo.

Talvez ali já estivesse nascendo aquilo que um dia iria se chamar Novo Caminho.

--- Eu chamo de \emph{Método Ars Moriendi} --- Roberto falou gesticulando. À sua frente estava três frascos, com agulhas.

--- E depois de seu teste? Você vai querer fazer isso?

--- Eu ainda não sei --- Roberto respondeu olhando para a palma de sua mão. ``Eu tenho medo do que eu posso encontrar do outro lado'', pensou em dizer. ``Se tiver um outro lado''.

Eles realizaram o primeiro experimento em uma sexta-feira, no dia onze de novembro. Eles resolveram deixar o lapso entre a linha plano do \textsc{ecg} e a ressuscitação bem curto, por volta de vinte e três segundos. E mesmo sendo um tempo relativamente curto, Patrick sentiu como se ele nunca fosse acabar.

--- Afaste-se! --- Roberto disse antes de aplicar trezentos joules, trazendo Arthur de volta à vida. Ele voltou fazendo um som estranho, como se tivesse repentinamente sentido uma dor profunda. Seus olhos brilhavam e fitavam o infinito.

Patrick e Roberto se olharam.

--- Estamos fazendo história --- Roberto disse sentindo uma excitação profunda. Ele estava feliz. Foi quando ele finalmente entrou na quinta e última fase do Modelo~Kübler-Ross, a \mudanca{aceitação}. ``Tudo vai ficar bem'', ele pensou.

--- Qual é a sensação de morrer? --- perguntou Patrick à cobaia que abria os olhos lentamente,

Arthur não respondeu de imediato. Ele estava sentido cada músculo de seu corpo.

--- Dói pacas --- ele disse por fim, soltando um suspiro. Os três riram.

E este foi apenas o começo. Logo Roberto e Arthur entraram em contato com um psicólogo, Edgar, autor de um livro sobre experiências de quase morte chamado \emph{Mors ontologica}.

--- Eu li o seu livro --- falou Roberto no primeiro encontro com Edgar, que tinha viajado de longe apenas para ver os resultados de Roberto.

--- O que achou?

--- Ótimo.

--- Quando você me enviou o e-mail eu achei que era alguma espécie de trote ou fantasia. Eu tive um pouco de receio. Mas é mesmo verdade --- Edgar dizia isso em um tom sério olhando para o vídeo. --- Isso é\ldots\,Legal? Judicialmente legal?

Roberto simplesmente deu de ombros. Os olhos não saíam um minuto sequer do monitor em que Roberto mostrava o vídeo da experiência com Arthur.

--- O que levou o senhor a fazer isso? --- perguntou Edgar olhando nos olhos de Roberto, que fugiram ao serem fitados.

--- Se tem algo, que você mesmo diz no livro, que é comum a todas as \textsc{eqm}s\mudanca{,} é que aqueles que passam por elas se transformam e mudam suas vidas, positivamente. Eu já li diversas filosofias, livros de autoajuda. Nós sabemos tudo o que deveríamos saber para viver uma vida plena e feliz. Mas não o fazemos. Por que somente pessoas que passam por experiências em que perdem tudo ou quase morrem têm a motivação necessária para fazer isso? Talvez algum defeito de fábrica, mas não me importa. Eu acho isso fascinante. Conseguir um propósito para vida. Arthur, o paciente zero, tem reportado resultados fantásticos. Foi isso que me levou a fazer isso.

Não demorou muito para que Edgar criasse coragem e perguntasse:

--- Roberto\ldots\,Eu poderia ser o próximo?

Eles começaram a se tornar amigos. Passavam noites em claro discutindo não apenas o \emph{Método Ars Moriendi}, explicações para as experiências de quase-morte, mas igualmente suas vidas, livros e tudo mais. Roberto, Arthur e Edgar haviam se tornado uma espécie de irmandade. Patrick passava muito tem\-po com eles, mas de vez em quando se afastava. Ele não suportava a atmosfera dos assuntos dos três que gravitavam ao redor da morte. Era uma obsessão.

--- Eu tenho um \mudanca{\emph{hobby}}. \mudanca{Coleciono} frases célebres --- falou Edgar\mudanca{,} certa vez. --- Há várias categorias\mudanca{,} e uma dessas são últimas palavras.

--- Eu gosto de frases --- disse Roberto tomando mais uma vez de sua garrafa.

--- Eu sei o que Nietzsche disse --- falou Arthur. --- Algo como, se existir um Deus vivo eu estou ferrado.

--- Ele resolveu fazer uma gracinha antes de morrer? --- perguntou Roberto. --- Nietzsche era um fanfarrão.

--- Uma das minhas prediletas é de Thomas Hobbes --- disse Edgar.

--- O sujeito da tábula rasa? --- questionou Roberto sem saber ao certo do que falava.

--- Não. Esse aí é Locke --- falou Edgar. --- Hobbes foi o que escreveu \emph{Leviatã}. Quando ele morreu, ele disse: ``Essa é a minha última viagem, um grande salto no escuro".

--- É realmente um grande salto --- \mudanca{disse Arthur, em tom confidente, a} Edgar.

--- Eu fiz teatro na faculdade. Antes de ser interno\mudanca{, quando} tinha tempo para ter uma vida --- falou Roberto rindo e pesaroso de se lembrar do passado. --- Eu li muito Shakespeare. \mudanca{Tem} uma parte que Hamlet fala sobre a morte e a compara a uma terra não-descoberta, da qual viajante nenhum jamais retorna. E hoje eu posso dizer que Shakespeare errou. Nós vencemos a morte, um grande soco na face de Deus --- concluiu com um sorriso no rosto.

--- Pena que não existe uma agência de viagens para essa terra não-descoberta --- falou Edgar apenas pensando em fazer uma observação espirituosa. --- Seria um grande sucesso, não acham?

--- ``Vamos para Ibiza esse ano, querido?'' --- Arthur falou imitando uma voz aguda, logo mudando para a grave: --- ``Não, não, mulher. Nós vamos morrer''.

Foi naquele momento que nasceu a Novo Caminho.
